% !TEX encoding = UTF-8 Unicode 
% !TEX root = praca.tex

\chapter{Conclusion}
% Intro
The social aspect of learning is a unique and important feature of the human species. The processes behind it are clarified by social learning theory and social epistemology. So far, social learning has been mainly studied using mathematical simulation models for static and synthetic communities. However, it is known that the real human communities have a unique and dynamic character that is difficult to capture by such models. \par
The presented thesis shows how the dynamic structure of a real community can influence the social learning process. It is important because the temporality of the structure of human communities is their immanent feature, which directly affects the processes taking place in them. Taking into account the temporal dimension of community evolution allows for a better and more complete representation of the real dynamics of group opinion formation. \par
% Literature, concept
The work includes a literature review of modeling diffusion processes in social networks, in particular social learning processes, and of modeling temporal networks. Selected algorithms from both fields are described. The concept of the temporal network epistemology model, the author's modification of the network epistemology model, is introduced. The modified model allows simulating the social learning process for temporal networks. Using the model, a study of the effect of temporal social network structure on this process was conducted. The network constructed using the CogSNet method based on the empirical NetSense dataset was used for this purpose. \par
% Conclusions
This study confirmed that the temporality of the social network has a significant impact on the course and outcome of the social learning process. The probability of obtaining correct consensus among temporal social network individuals is lower than in static models, and the process of opinion formation is slower. The temporal social network is also the only structure for which the average course of the process is not stably increasing, but fluctuations and oscillations occur, reflecting changes happening in the community. \par


\section{Future Work}
The proposed temporal network epistemology model can be considered as a first step towards studying learning processes in temporal social networks. The research presented here explores only a small fraction of the possibilities for using the model to better understand how human communities collectively work to solve the problems set before them. Several of the possible future directions for this work are outlined below. \par

\subsection{Social Phenomena}
In this work, the proposed temporal network epistemology model is based on the basic learning model proposed in \cite{zollman_2011}. In recent years, there have been several noteworthy modifications of this model proposed in \cite{oconnor_weatherall_2018}, \cite{oconnor_weatherall_2018_2}, \cite{oconnor_weatherall_2018_3}, \cite{oconnor_weatherall_2020_2}, which are further described in the first chapter. These extensions enable to capture some social behaviors like conformity or social trust capital, resulting in social phenomena like polarization and propaganda, which are often observed in real human societies after all. Adding these extensions to the temporal network epistemology model could allow verifying the operation of such phenomena in dynamic networks. \par

\subsection{Experimental Data Expansion}
This study was conducted on only one temporal social network constructed from the NetSense dataset. In order to be able to draw stronger conclusions based on the results presented here, it would be necessary to repeat the experiments on a larger number of temporal networks. There are other reference empirical datasets with different characteristics that could be used for this purpose, such as \cite{chaintreau_2007}, \cite{michalski_2020_2}. Also worth considering is the use of synthetic temporal network models described in the second chapter that allow the generation of such structures. Such a study could confirm or reject the observations presented in this work and shed new light on the problem under study. \par

\subsection{Temporal Structure Synchronization}
From the perspective of temporal network structure, an important direction of development would be to verify the influence of communication and update frequency on the effectiveness and speed of the learning process. A potential experiment could be to test temporal social networks created for time windows of different resolutions. Showing how this process depends on the frequency of information exchange could help to differentiate communities that are better and which are worse prepared to cope with tasks from the category corresponding to the problem class like in the network epistemology model. \par

\subsection{Belief Spread Maximalization}
Another interesting research direction that could be worth exploring is the issue of social influence study. Similar studies have already been conducted for other models of diffusion in social networks \cite{michalski_2014}. More recent research indicates that the relevance of individuals on network spread can be effectively estimated using entropy-based measures \cite{michalski_2020}. A separate direction could also be to study the effect of initial conditions on the learning process. \par


\section{Summary}
As a conclusion, the thesis was successful in achieving the previously stated goal, and the research findings presented in this paper indicate the clear influence of temporal network structure on social network learning. Most people do not hold blindly to their beliefs, changing them under the influence of evidence presented and interactions with colleagues and friends. These very friendships, like the beliefs themselves, are not permanent. Social ties dynamically evolve over time and thus determine the process of social exchange of knowledge. This aspect should not be ignored when evaluating the potential of social networks in the context of solving epistemic problems. \par
