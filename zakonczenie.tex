% !TEX encoding = UTF-8 Unicode 
% !TEX root = praca.tex

\chapter*{Conclusion}
% Intro
The social aspect of learning is a unique and important feature of the human species. The processes behind it are clarified by social learning theory and social epistemology. So far, social learning has been studied using mathematical simulation models mainly for static and synthetic communities. However, it is known that real human communities have a unique and dynamic character that is difficult to capture by such models. \par
The presented thesis shows how the dynamic structure of a real community can influence the social learning process. It is important because the temporality of the structure of human communities is their immanent feature, which directly affects the processes taking place in them. Taking into account the temporal dimension of community evolution allows for a better and more complete representation of the real dynamics of group opinion formation. \par
% Literature, concept
The work includes a literature review of modeling diffusion processes in social networks, in particular social learning processes, and of modeling temporal networks. Selected algorithms from both fields are described. The concept of the temporal network epistemology model, the author's modification of the network epistemology model, is introduced. The modified model allows simulating the social learning process for temporal networks. Using the model, a study of the effect of temporal social network structure on this process was conducted. The network constructed using the CogSNet method based on the empirical NetSense dataset was used for this purpose. \par
% Conclusions
This study confirmed that the temporality of the social network has a significant impact on the course and outcome of the social learning process. The probability of obtaining correct consensus among temporal social network individuals is lower than in static models, and the process of opinion formation is slower. The temporal social network is also the only structure for which the average course of the process is not stably increasing, but fluctuations and oscillations occur, reflecting changes happening in the community. \par
% Future work
The paper outlines several potential directions for developing research based on the presented contribution. A key next step would be to validate the observations outlined in the paper for a larger number of temporal social networks built on empirical datasets. The proposed model can also be extended to account for social phenomena such as polarization or propaganda. \par
% Summary
As a conclusion, the paper was successful in achieving the previously stated goal, and the research findings presented in this paper indicate the clear influence of temporal network structure on social network learning. Most people do not hold blindly to their beliefs, changing them under the influence of evidence presented and interactions with colleagues and friends. These very friendships, like the beliefs themselves, are not permanent. Social ties dynamically evolve over time and thus determine the process of social exchange of knowledge. This aspect should not be ignored when evaluating the potential of social networks in the context of solving epistemic problems. \par
