% --- Strona ze streszczeniem i abstraktem --------------------

\chapter*{Abstract}
The human learning process would not be possible without social relationships that allow communities to exchange views and to learn through social interactions. The aim of this thesis is to study the impact of temporal network structure on the social learning process. Previous research in this area has focused on static networks which do not capture the dynamics of connectivity changes that are present in real social networks. This work develops the concept of temporal network epistemology model enabling the simulation of learning process in dynamic networks. The results of the research, conducted on both static topologies and a temporal social network generated using the CogSNet method, indicate a significant influence of the network temporality on the outcome and flow of the learning process. The model and the results of the conducted experiments can be used to better understand the way in which human communities can collectively solve both complex problems at the scientific level and inquire into the correctness of less complex but common and equally important beliefs spreading across entire societies. \par


\addtocontents{toc}{\protect\setcounter{tocdepth}{-1}}
\begingroup
\renewcommand{\cleardoublepage}{}
\renewcommand{\clearpage}

\chapter*{Streszczenie}
Proces uczenia się charakterystyczny dla ludzi nie mógłby zaistnieć, gdyby nie złożone zależności społeczne, pozwalające na wymianę wiedzy oraz społeczne uczenie się. Celem tej pracy jest zbadanie wpływu temporalnej struktury sieci na proces społecznego uczenia się. Dotychczasowe badania w tym obszarze skupiały się na wykorzystaniu sieci statycznych, które nie oddają dynamiki zmian, występującej w prawdziwych sieciach społecznych. W ramach pracy przygotowano koncepcję temporalnego modelu epistemologii sieciowej, pozwalającego na modelowanie procesu uczenia się w sieciach dynamicznych. Wyniki badań, przeprowadzonych zarówno na topologiach statycznych jak i sieci temporalnej wygenerowanej z wykorzystaniem metody CogSNet, wskazują na istotny wpływ dynamicznej struktury sieci na rezultat oraz przebieg procesu uczenia się w sieciach. Opracowany model oraz wyniki przeprowadzonych badań mogą być wykorzystane w celu lepszego zrozumienia sposobu, w jaki ludzkie społeczności są zdolne do kolektywnego rozwiązywania zarówno problemów na poziomie naukowym, jak i dociekania poprawności mniej złożonych, ale równie ważnych poglądów rozprzestrzeniających się w całych społeczeństwach. \par

\endgroup
\addtocontents{toc}{\protect\setcounter{tocdepth}{2}}
% --- Koniec strony ze streszczeniem i abstraktem -----------------------------------------------------------
