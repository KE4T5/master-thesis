% --- Strona ze streszczeniem i abstraktem --------------------

\chapter*{Abstract}
The human learning process would not be possible without complex social relationships that allow human communities to exchange views and knowledge and to learn through social interactions. The aim of this thesis was to study the impact of temporal social network structure on the social learning process. Previous research in this area has focused on static network structures which do not capture the real dynamics of connectivity changes that are present in real social networks. This work develops the concept and implementation of temporal network epistemology model enabling the simulation of learning process in dynamic networks. The results of the research, conducted on both static topologies and a temporal social network generated using the CogSNet method from the NetSense dataset, indicate a significant influence of the network temporality on the outcome and flow of the learning process in social networks. The developed model and the results of the conducted experiments can be used to better understand the way in which human communities are capable of collectively solving both complex problems at the scientific level and inquiring into the correctness of less complex but common and equally important beliefs spreading across entire societies. \par


\addtocontents{toc}{\protect\setcounter{tocdepth}{-1}}
\begingroup
\renewcommand{\cleardoublepage}{}
\renewcommand{\clearpage}

\chapter*{Streszczenie}
Proces uczenia się charakterystyczny dla ludzi nie byłby możliwy do zaistnienia, gdyby nie złożone zależności społeczne, pozwalające ludzkim społecznościom na wymianę poglądów i wiedzy oraz społeczne uczenie się. Celem tej pracy było zbadanie wpływu temporalnej struktury sieci społecznej na proces społecznego uczenia się. Dotychczasowe badania w obszarze tej problematyki skupiały się na wykorzystaniu struktur sieci statycznych, które nie oddają rzeczywistej dynamiki zmian połączeń występującej w prawdziwych sieciach społecznych. W ramach pracy przygotowano koncepcję oraz implementację temporalnego modelu epistemologii sieciowej, pozwalającego na modelowanie procesu uczenia się w sieciach dynamicznych. Wyniki badań, przeprowadzonych zarówno na topologiach statycznych jak i temporalnej sieci społecznej wygenerowanej z wykorzystaniem metody CogSNet ze zbioru danych NetSense, wskazują na istotny wpływ dynamicznej struktury sieci na rezultat oraz przebieg procesu uczenia się w sieciach społecznych. Opracowany model oraz wyniki przeprowadzonych badań mogą być wykorzystane w celu lepszego zrozumienia sposobu, w jaki ludzkie społeczności są zdolne do kolektywnego rozwiązywania zarówno skomplikowanych problemów na poziomie naukowym, jak i dociekania poprawności mniej złożonych, ale powszechnych i równie ważnych poglądów rozprzestrzeniających się w całych społeczeństwach. \par

\endgroup
\addtocontents{toc}{\protect\setcounter{tocdepth}{2}}
% --- Koniec strony ze streszczeniem i abstraktem -----------------------------------------------------------
