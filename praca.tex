% !TEX encoding = UTF-8 Unicode 

\documentclass[magister,druk]{dyplom}
%	\usepackage[cp1250]{inputenc}
%utf8
\usepackage{hyperref}
\usepackage{amssymb}
\usepackage[figuresleft]{rotating}
%%
\usepackage[tfoc]{appendix}
\renewcommand{\appendixtocname}{Dodatki}
\renewcommand{\appendixpagename}{Dodatki}

% Głębokość numerowania sekcji /section /subsection /subsubsection ...
\setcounter{secnumdepth}{4}

% pakiet do składu listingów w razie potrzeby można odblokować możliwość numerowania linii lub zmienić wielkość czcionki w listingu
\usepackage{minted}
\setminted{breaklines,
frame=lines,
framesep=5mm,
baselinestretch=1.1,
fontsize=\small,
%linenos
}

% nowe otoczenie do składania listingów
\usepackage{float}
\newfloat{listing}{htp}{lop}
\floatname{listing}{Listing}
\usepackage{chngcntr}
\counterwithin{listing}{chapter}

% patch wyrównujący spis listingów do lewego marginesu 
%https://tex.stackexchange.com/questions/58469/why-are-listof-and-listoffigures-styled-differently
\makeatletter
\renewcommand*{\listof}[2]{%
  \@ifundefined{ext@#1}{\float@error{#1}}{%
    \expandafter\let\csname l@#1\endcsname \l@figure  % <- use layout of figure
    \float@listhead{#2}%
    \begingroup
      \setlength\parskip{0pt plus 1pt}%               % <- or drop this line completely
      \@starttoc{\@nameuse{ext@#1}}%
    \endgroup}}
\makeatother

\usepackage{url}
\usepackage{lipsum}

% Dane o pracy
\author{Damian Serwata}
\title{Dynamika sieci społecznej, a dynamika rozprzestrzeniania się informacji}
\titlen{The dynamics of social network compared to the dynamics of the spread of information}
\promotor{dr inż. Radosław Michalski}
\wydzial{Wydział Informatyki i Zarządzania}
\kierunek{Informatyka}
\specjalnosc{Danologia}

\krotkiestreszczenie{This work explores a network epistemology model for a temporal social network constructed using the CogsNet model.}
\slowakluczowe{network epistemology, social networks learning, temporal networks, CogSNet model}

\begin{document}

\maketitle

\tableofcontents

% --- Strona ze streszczeniem i abstraktem --------------------

\chapter*{Abstract} % ...i to samo po angielsku
The human learning process would not be possible without complex social relationships that allow human communities to exchange views and knowledge and to learn through social interactions. The aim of this thesis was to study the impact of temporal social network structure on the social learning process. Previous research in this area has focused on static network structures which do not capture the real dynamics of connectivity changes that are present in real social networks. This work develops the concept and implementation of temporal network epistemology model enabling the simulation of learning process in dynamic networks. The results of the research, conducted on both static topologies and a temporal social network generated using the CogSNet method from the NetSense dataset, indicate a significant influence of the network temporality on the outcome and flow of the learning process in social networks. The developed model and the results of the conducted experiments can be used to better understand the way in which human communities are capable of collectively solving both complex problems at the scientific level and inquiring into the correctness of less complex but common and equally important beliefs spreading across entire societies. \par

% Kilka sztuczek, żeby:
% - Abstract pojawił się na tej samej stronie co Streszczenie
% - Abstract nie pojawił się w spisie treści
\addtocontents{toc}{\protect\setcounter{tocdepth}{-1}}
\begingroup
\renewcommand{\cleardoublepage}{}
\renewcommand{\clearpage}

\chapter*{Streszczenie} % po polsku
Proces uczenia się charakterystyczny dla ludzi nie byłby możliwy do zaistnienia, gdyby nie złożone zależności społeczne, pozwalające ludzkim społecznościom na wymianę poglądów i wiedzy oraz społeczne uczenie się. Celem tej pracy było zbadanie wpływu temporalnej struktury sieci społecznej na proces społecznego uczenia się. Dotychczasowe badania w obszarze tej problematyki skupiały się na wykorzystaniu struktur sieci statycznych, które nie oddają rzeczywistej dynamiki zmian połączeń występującej w prawdziwych sieciach społecznych. W ramach pracy przygotowano koncepcję oraz implementację temporalnego modelu epistemologii sieciowej, pozwalającego na modelowanie procesu uczenia się w sieciach dynamicznych. Wyniki badań, przeprowadzonych zarówno na topologiach statycznych jak i temporalnej sieci społecznej wygenerowanej z wykorzystaniem metody CogSNet ze zbioru danych NetSense, wskazują na istotny wpływ dynamicznej struktury sieci na rezultat oraz przebieg procesu uczenia się w sieciach społecznych. Opracowany model oraz wyniki przeprowadzonych badań mogą być wykorzystane w celu lepszego zrozumienia sposobu, w jaki ludzkie społeczności są zdolne do kolektywnego rozwiązywania zarówno skomplikowanych problemów na poziomie naukowym, jak i dociekania poprawności mniej złożonych, ale powszechnych i równie ważnych poglądów rozprzestrzeniających się w całych społeczeństwach. \par

% Wprowadzenie
%Celem pracy było opracowanie aplikacji służącej do komunikacji z kosmitami. Dostępne na rynku aplikacj e nie satysfakcjonowały autorki ze względu na brak istotnych funkcji takich jak obsługa przez telefon z systemem Android.
% Sposób rozwiązania problemu
%W ramach pracy przygotowano aplikację komunikacyjną wykorzystującą framework SpaceDirect, przechowującą dane kontaktów w bazie danych MyNoSQL oraz udostępniającą swoje funkcje przez interfejs REST API.
% Dodatkowe informacji o pracy
%Oprócz projektu aplikacji praca zawiera wyniki testów jednostkowych oraz testów użyteczności przeprowadzonych przez krewnych i znajomych królika.
% Podsumowanie
%Przygotowana w ramach projektu inżynierskiego praca może zostać wykorzystana przez wszystkie osoby zainteresowane kontaktami z cywilizacjami pozaziemskimi.


\endgroup
\addtocontents{toc}{\protect\setcounter{tocdepth}{2}}
% --- Koniec strony ze streszczeniem i abstraktem -----------------------------------------------------------


% - Here I analyze network epistemology model in which agents, all else being equal, prefer to take actions that conform with those of their neighbors. 

% !TEX encoding = UTF-8 Unicode 
% !TEX root = praca.tex

\chapter*{Introduction}

% Wprowadzenie w tematykę

% 1. Społeczny aspekt komunikacji wśród ludzi
Humans are extremely social animals. The specific abilities that our species has developed in the course of the cognitive revolution have allowed humans to dominate other organisms on our planet[Harari, Yuval Noah. 2015. Sapiens: A Brief History of Humankind.]. One crucial element in this process is the incredibly powerful and effective, evolved ability of homo sapiens to communicate and exchange large amounts of information about social relationships and the world around us. Given the evolutionary history of our species, it should come as no surprise that human systems for forming views and opinions are also social in nature. Humans pass information from one person to another, and as a result a large number of our views are dependent on the environment of our contacts. It is this evolutionarily evolved and inherent human way of spreading knowledge that is the foundation of today's extremely complex and advanced culture and science.

% 2. Modelowanie
Due to the above-mentioned role of the social aspect of information exchange in the social influence process, researchers have long been carrying out research that helps to understand the characteristics of this process and its complexity. There is a growing interest in modelling in the scientific community. What is the best community structure for spreading information? What conditions must be met to allow the community to reach consensus on a given topic? What influences the speed of the spread of information and opinions? Various models developed by researchers in recent years provide answers to these questions. Through the use of mathematical modelling it is possible, to a certain extent, to predict spreading scenarios and to gain insight into the complex dynamics of the social influence process. This research is extremely important and useful, but often the experiments are conducted on model network structures. Therefore, the results of such research should always be approached with caution and, if possible, verified.

% 3. Network Epistemology Model
Many of these models are based on the functioning of social networks, that have been identified as a major factor in determining how societies move towards adopting ideologies. One social network model that provides an interesting way of capturing aspects of social learning in human societies and its impact on theory adoption is the Network Epistemology Model. It is an agent-based model that offers an imitation of the human and especially scientific way of analysing information and updating beliefs. It has so far mainly been used to model the spread of knowledge in scientific communities, but it can also be applied in a more general context, to study any community of people receiving evidence and sharing their beliefs.

% 4. Temporalna natura sieci społecznych
An important feature of the social networks studied by the researchers is their temporality - acquaintances and friendships appear and disappear, old relationships are forgotten, but new people are met. Most real networks are temporal and their structure - the confluence of nodes and edges - is changing over time. The temporal structure of link activity can influence the dynamics of interaction in a network just like its topology. Researchers have already proposed many models, which attempt to capture this aspect of social networks in different ways. The novel cognition-driven social network (CogSNet) model proposes unique approaches based on social perception.\\

% Co zawiera praca + odeniesienie do tytułu
This work addresses the topic of modelling social influence in temporal social networks. The question is whether the models that are currently used in the domain correctly represent the dynamics of opinion propagation in real social networks. Special attention is given to the Network Epistemology Model, for which so far no extension has been developed to make the model work for temporal networks, and to the CogSNet temporal network model.

The dynamics of the social network mentioned in the title of the paper is understood here as the temporality of the network structure, and the dynamics of the spread of information as the process of social influence - the spread of views and theories in communities.



\section*{Objective}

The main aim of the thesis was to investigate the influence of the temporal nature of real-world networks on the process of information diffusion and, in particular, process of learning and social influence in social networks.

The main purpose of this paper is to analyse how the results of one of the typical models of spread of influence <model> differ depending on the strategy of building the social network used later for ...


\section*{Scope of work}
% TODO: name of telco dataset
The work involved developing a concept and implementing a modification of the Network Epistemology Model, in a way that allows the model to work for temporal networks. Implementation of the basic model and the extension was made in Python programming language. The structure of the real temporal network was generated using the CogSNet model and a set of interactions between students using instant messaging applications on smartphones from TelcoData dataset. Using the developed model and the generated temporal network structure, a series of studies were performed including an analysis of the influence of the temporal structure of the network on the consensus formation process and its characteristics.\\

The paper is divided into five chapters. The first and second chapter presents a review of the literature on social influence modelling in temporal social networks. In the third chapter, the Network Epistemology Model is presented in detail, as well as the author's concept of modifying this model to make it work in temporal networks. The fourth chapter focuses on the description of the methodology and research conducted using the model. In the fifth chapter there is an analysis of the obtained results and a discussion about possible further research using the model.

% -------------------------------------------


% Dodatkowe
% Illustrate how the social aspects of science can influence theory adoption and lead to potential problems.
% Celem pracy jest opracowanie modelu Model ma umożliwiać analizę zjawiska narzędzie analizy oraz badanie wpływu różnych czynników na jego zasięg i dynamikę

% !TEX encoding = UTF-8 Unicode 
% !TEX root = praca.tex


\chapter{Social Networks and Learning}

\section{Introduction}
This chapter will discuss the topic of learning in social networks. Key concepts will be presented as well as questions that researchers studying this topic are trying to answer. It will then list popular and important models used over the years to study the spread of information and beliefs.


\section{Issues and Applications}
% Social Systems 
Social systems are extremely difficult to describe and explain, despite developments in statistical models and computing power. What we observe in these systems is the effect of the influence of a huge number of internal and external factors. The internal ones include the flow of information between the participants of social life, the lack of knowledge of the initial conditions of the microscopic state of systems, and the precisely defined laws governing them or, more generally, social laws. External factors, in turn, include political events, natural factors (disasters, climatic phenomena), and technological factors, as well as the influence of other societies. Moreover, the individuals, that form the communities, are intelligent beings, who can learn and adapt their behavior to changing circumstances. The enormous number of these factors and their interconnectedness makes human societies some of the most complex systems existing in the Universe. All this together means that a correct and comprehensive description of these communities requires taking into account the information available at different levels of society. The need for such a holistic description is present also for many other complex systems in nature. \par
% Networks and social networks intro
Networks are the data structures that allow to characterize and study many complex systems such as infrastructure grids, ecosystems, climate, but also people societies. A social network is a specific type of complex network whose structure consists of actors (people, organizations) as nodes, and social interactions between the actors as links \cite{green_bossomaier_2000}. Social networks analysis is a vast and interdisciplinary academic field with origins in sociology, social psychology, graph theory, and statistics. It focuses on the study of relationships between individuals, groups of individuals, and even entire societies. \par
% Learning in networks
The formation of opinions and sharing the information between individuals is determined by their social networks. Obtaining information and consulting opinions with other humans is an evolutionary trait, that gives humans a huge advantage over other animals that do not have this ability. People tend to consult their friends and neighbors for opinions over a new firm they want to apply for, discuss the credibility of political parties, or just ask what movie to watch. A good understanding of the way that social networks influence the process of opinion formation and sharing is critical for understanding these phenomenons themselves.
One of the most important sections of social network analysis is the study of information diffusion and learning in social networks. One of the most common ways to disseminate new ideas and opinions is to learn from friends and neighbors. Most social learning processes have two distinct components. First, is the exchange of ideas between people who have a made opinion about something. Second, is the spread of new information to a person unfamiliar with the subject. Social relationships allow to information transmission by observing other people's decisions and by talking and sharing opinions. Understanding how individuals use information from their social environments and the implications of this process is important in many contexts. The key role played by the transmission of this information has been demonstrated for various phenomena such as product selection \cite{trusov_2008}, voting in elections \cite{allen_2002}, and job search \cite{montgomery_1991}. It is well known that the characteristics of contacts configuration, in a form of network structure, influences many aspects of social life. Among other things, the studies conducted so far have proven that even a community of perfectly rational individuals can come to incorrect conclusions. People who alone would have no problem distinguishing truth from falsehood can form irrational groups \cite{zollman_2013}. \par
% Questions
Some of the important questions regarding the formation of the beliefs in social networks are:
\begin{itemize}
    \item Which individuals have the biggest influence over the beliefs in a whole society?
    \item Whether individuals will conform?
    \item Whether individuals in society come to the right belief?
    \item Whether dispersed information is aggregated accurately and efficiently?
    \item How long it takes for the individuals learn?
\end{itemize} \par
% Ending
The social learning process is distributed through space and time. It usually requires the involvement of up to thousands of people. Theories adaptation may take many years and numbers of interactions between the people. This is why observational and experimental methods are not sufficient for studying large-scale social effects. The motivation for simulation and modelling is the complexity of social systems of non-linear, multi-level, and dynamic interactions. Modelling different types of social functions is often not feasible by other computational methods. Computer simulations and modelling are extremely useful in case when it is hard to investigate a community of people empirically. Testing hypotheses, about which factors might make a difference to how such groups of people learn is possible by building simple programs that can mimic groups of people sharing views. The results of such simulations can help to interpret what we experience in the real world - and even suggest new ways of seeing the full complexity of human societies. Analysis of how these models work, confirmed by empirical studies can change our fundamental understanding of ourselves.



\section{Models and Approaches}
% Computer simulations
Support of social sciences by mathematical models as well as statistical and computational methods has been used successfully for a long time. All this is done in order to describe social phenomena, which are among the most complex in nature. In recent years, mathematical models are more and more effectively used to describe phenomena observed in complex systems. These systems, because of their complex mathematical description (significantly related to their nonlinearity, which prevents obtaining accurate results of analytical calculations), can be studied mainly by approximate or numerical methods, observing the time evolution of their states. As a result of the development of computer techniques, many works have been proposed describing systems, whose structure and dynamics have so far been based on quantitative analysis. Mathematical description enables a better understanding of the processes happening in complex networks \cite{holme_saramaki_2012}. An important element of such research, is the cognitive aspect, as complex networks are commonly found in society. The phenomena occurring inside them have, on the one hand, a system-specific character, while on the other hand, in many cases they have a universal character, showing similarity to phenomena occurring in other natural systems. \par
% Historycznie
The first studies of social learning were concentrated on social influence topics and explored the theory of opinion leaders. These focused on the question of which individuals tend to become opinion leaders. Subsequently, scientists have consistently developed more complex and realistic models that have made it possible to answer the wider and wider range of questions presented in the previous section. For years, attempts have been made to create models that reproduce changes in community beliefs over time. Although it has been a problem to relate many theoretical results to real data, the situation has changed in recent years thanks to data obtained from social media networks. So far, a very large number of opinion dynamics models have been developed. Among them, there are some simple fundamental models that allow us to understand the most important phenomena and capture the overall sense of the problem. \par
% Klasyfikacja modeli
Models can be divided into two main, fundamentally different types: macroscopic and microscopic. The first ones help to answer the questions of how? and how much? They do not represent what happens at the microscale, at the level of individual units of analysis. They model the behaviour of appropriate average values. The main disadvantage of such models is the lack of an answer to the question about the causes of occurring phenomena. Microscopic models are created in order to try to answer the question of why something happens. \\
In the case of the economic and social sciences, models of the following type are distinguished:
\begin{itemize}
    \item Microsimulation, where changes in the states of objects are determined by certain deterministic or stochastic rules.
    \item Agent Based Models, in which the system under study is a set of agents that interact with each other according to certain model-dependent rules.
\end{itemize} \par
% Agent Based Modelling
A frequently used approach is agent based modelling, in which a single agent is equivalent to a person and the whole system consists of a network of such agents. An agent, as the basic element of a system, has certain numerically described characteristics like opinion and usually interacts with other agents or external factors.  Both the characteristics of a single agent and the rules of interaction are dependent on the specific model. An opinion is a numerical representation of an agent's belief, in most cases, it can be an integer, a continuous number, or a vector. \\
The general scheme in agent based models can be presented as follows:
\begin{enumerate}
    \item Single agents or groups of agents are arranged - the algorithm for selecting agents for meetings depends on the model.
    \item A discussion follows, as a result of which agents can change their opinion - here again, the details of the interaction depend on the specific model.
    \item The procedure is repeated.
\end{enumerate} \par
% Bayesian or myopic
Other criteria by which the different approaches can be divided are whether learning is Bayesian or myopic and whether individuals learn from specific signals or by observing actions taken by others. The advantage of Bayesian updating is its solid normative basis but it's often complex. The observation and updating of views in this model are done in a sophisticated way. Alternatively, simpler models may not pay attention to the actions of specific individuals, only observe their opinion and update their own by averaging. \\
The following subsections will discuss several types of different models that have so far been frequently used to study the problem of social learning.


\subsection{Epidemic Models}
Epidemic models have been used mainly to predict the spread of diseases and the development of epidemics. However, it has also been used to model the diffusion of information. In these models, concepts or views are treated as viruses that spread between individuals within a social network. There are different ways in which such an infection can spread. Some models assume that any individual who neighbours an infected one will also become infected themselves. In others, however, ideas spread whenever a certain percentage of neighbours become infected. \par
The most popular approach to epidemic modelling is Susceptible-Infectious-Recovered (SIR), in which each individual can be in one of 3 states. Many modifications of this model have been developed that take into account births and deaths, lack of immunity after recovery (SIS) or temporal immunity (SIRS) \cite{bailey_1975}. Some studies have shown that in some specific situations epidemic models can correctly model the spread of views and ideas.
These models have been used successfully to study the propagation dynamics of topics \cite{guha_2004}. 


\subsection{Sznajd Model}
The concept of the Sznajd model \cite{sznajd_2001} is based on a fact known in social psychology as social proof. If a person with an opinion that differs from ours tries to convince us that he is right, he has a difficult task ahead of him. But when, at the same time, two or more people agree with each other, they can succeed incomparably easier. Based on this insight, a model of opinion dynamics was developed and proved to be a huge success. It is a simple but yet important variation of the prototypical Ising system. It is still studied today and continues to provide many interesting results. \par
In the original version of the model, simulations were performed on a closed one dimensional chain. Agents can take only one of two possible opinions -1 or +1. In a simulation, groups of agents are arranged to meet, discuss and change current opinion. Such dynamics lead to a so-called consensus, i.e. a state in which all agents share the same opinion. There are two possible states in this model. First is social validation - when people share the same opinion, then their neighbors start to agree with them. Second is discord destroys - when connected individuals disagree, then their neighbors start to argue with them too.
This model, with modifications, has been applied to explain the process of opinion exchange in finance, politics, and marketing, among other fields \cite{sznajd_2005}.


\subsection{Threshold Models}
Threshold models describe how individuals decide to change their views or behaviour by following the behaviour of their friends \cite{granovetter_1978}. These models can be used to analyse a wide range of phenomena such as the spread of information, rumours and social influence. Individuals change their views when a certain percentage of their neighbours have done so too. Depending on the type of relationship and its importance, different neighbours may have a different role in convincing a particular individual to change his decision. \par
The most popular of the threshold models used for the social influence problem is the Linear Threshold Model \cite{granovetter_1978}. In this model, individuals are connected by relationships with different weights, that characterise the strength of the interaction between neighbours. Each of them is also assigned a threshold representing the boundary value beyond which a change of view occurs. The process ends when none of the individuals influences anyone else anymore. Threshold values assigned to specific individuals can be initialised by drawing from some probability distribution or initialised as fixed values.
Threshold models found applications in modeling diffusion of innovations \cite{valente_1996} and spread of social influence \cite{kempe_2003}.


\subsection{Binary Agreement Models}
The binary agreement model also called a naming game model is another approach to modelling the spread of opinions. The process is iterative. In each iteration, a speaker and a listener are drawn at random and linked by a relationship. Then they communicate with each other, exchanging their opinions. In this model, each individual is assigned one of two competing views or both opinions. During communication, if the listener already holds the view presented by the speaker, they both retain it and discard the opposing view. Otherwise, the listener updates their set of views with the newly presented one. \par
Research conducted by scientists has shown that this model is able to guarantee consensus for individuals organised in a complete graph structure. The time in which this happens depends on the number of agents in the network \cite{castello_baronchelli_vittorio_2009}. A very interesting property of this model is also the fact that individuals promoting their own views seek a broader consensus \cite{kearns_2009}. Using this model, it has been shown, that even a small proportion of individuals resistant to change of opinion and constantly proclaiming their fixed views, can be sufficient to convince an overwhelmingly large proportion of the community, to adopt a new view \cite{szymanski_2011}.


\subsection{Naive Learning Models}
% Naive learning
These models are based on the assumption, that individuals do not perform particularly complex calculations and analyses, when updating their opinions. Naive learning, also known as DeGroot learning, works on the principle that individuals periodically communicate with their neighbors and average their opinions in the simplest possible way. This myopic model is more advanced than simple models of information propagation but still uncomplicated. Despite this relative simplicity, it has proven capable of adequately modeling the behavior of real human communities. This model is considered to be one of the canonicals for representation of opinion propagation processes, due to its characteristics. An important property in this context is that once certain known assumptions are met, a consensus is guaranteed to be reached in a finite time \cite{banerjee_2019}. It has been widely applied in many fields such as economics, sociology, statistics, and politics \cite{demarzo_2003}. \par
% De Groot model
In the DeGroot model, agents are connected in a weighted directed network describing the directions of information flow between individuals. Agents start the learning process with some fixed views. Then, in an iterative process, each individual establishes its view as an average of its previous opinion and that of its neighbors. Over time, the opinions of all individuals align and, assuming certain criteria are met, the network reaches consensus \cite{jackson_2008}. \par
% Properties, other
This model simulates some phenomena characteristic of human communities in a very interesting way. The own opinion of individuals can act as an echo, and come back to them, increasing their confidence in the stated theory. Another example is the repeated counting of opinions of individuals sharing a larger number of common neighbors. The biggest influence on the spread of an individual's opinion is his centrality in the network structure.
The final consensus depends on the initial views and the value of the centrality measure of the nodes in the network. This model makes it extremely simple to compute the state of views in a network at any point in time, using matrix computation. The formula boils down to summing the initial opinions of each individual multiplied by its eigenvector centrality. The conditions for achieving consensus in the DeGroot model are a diversity of initial views that cannot be systematically biased and a balanced structure of connections in the network. The eigenvector centrality of each individual in the network should be relatively small, compared to the sum of the centralities of the rest of the agents. In other words, no small group of agents can have a monopoly on proclaiming their views \cite{jackson_golub_2010}. In social networks, where some individuals have too much influence on the rest of the community, due to a large number of contacts, it may be impossible to carry out a correct learning process. \par


\subsection{Observational Learning Models}
% Observational learning
Observational learning models are characterized by the fact, that individuals within the network structure can observe each other's actions and the results they achieve. This group of models is, by the nature of its operation, most similar to the philosophical theories of social choice and social epistemology. These theories focus on the issue of aggregating individual views into group beliefs. Working together is better than going it alone because when faced with a problem to solve, alone people tend to get attached to wrong theories too quickly. \par

% Network Epistemology Model
Representative of this category of models is the network epistemology model, which is based on ideas from many fields like biology and economics combined to explain learning in epistemic communities. Epistemic communities are special cases of social communities, whose members strive to gain knowledge about the world around them. An example of such a community might be a community of scientists in some field. Such communities of scientists have often been the focus of researches performed using this model, but it has a much broader application. It can be used to model any community of people gathering evidence from their environment and exchanging beliefs \cite{missinformation_age}.
This is because all people, both individually and in communities interact with the outside world in different ways. Depending on the results of these interactions, they also adjust their perspective on the world and their strategies of acting. It is in this way - trying to understand the world on the basis of past experience - that people act in a scientifically similar way, though certainly less precisely and systematically.
In this model, unlike simpler ones such as epidemic models, for example, beliefs do not simply transfer from one person to another. Each individual has some degree of confidence in a particular view, which leads him to collect evidence to support it. This evidence influences the subsequent verification of the views. Each agent shares the collected evidence with his neighbors, which also influences the formation of their opinions. \par

\begin{figure}
\centering\includegraphics[width=.8\textwidth]{img/rozdzial1/epinet_example.png}
\caption{An example of an opinion spread in an epistemic network. (Source: \cite{sep-epistemology-social})}  \label{rys:network}
\end{figure}

% Bala & Goyal
The network epistemology model was initially proposed in \cite{bala_goyal} to model the process of knowledge propagation. In this model, agents are connected in an undirected network. The learning process is iterative. The problem that agents face includes choosing between two options. Agents have certain beliefs about which action is better. At the beginning of the simulation, agents are assigned random confidence levels about the effectiveness of particular actions. In each iteration, the agents choose the action about which they have the highest confidence in its success. The agents do not know the expected payoff value of the action. Their knowledge is based only on observing the results returned by the actions, which are random in nature. After executing the chosen action, individuals update their views using Bayes' rule based on their own and their neighbors' observations. The simulation ends when the network converges to one of the views. In this most basic version of the model, it means that the communities in question come to a correct or incorrect consensus.
Using this model, it has been observed that even a community of perfectly rational people, trying to figure out the truth about the world, can come to false conclusions. This is because the actual scientific evidence obtained by interacting with the world is probabilistic. For example, not all cigarette smokers will develop lung cancer. By observing too small a control group, it may be that none of the smokers get sick. Such cases, though obviously less likely, do occur and are the genesis of studies showing no statistical relationship between smoking and cancer. Such a misleading study, if sufficiently well publicized, may be enough to mislead the entire community. \par

% Zollman
Subsequently, work on this model was undertaken by philosopher Kevin Zollman. It was he who proposed using the model to mimic the operation of scientific communities. His work shows that the model can help understand the consensus building process in such communities. At first glance, it would seem that better communication of experimental results between individuals should improve community learning. It is true that such more densely connected communities come to a consensus more quickly, but more often than not it is an incorrect consensus. The results of his simulations show that the structure of a social network has a direct impact on the likelihood of reaching a correct consensus \cite{zollman_2011}. In particular circumstances, fewer connections and thus a slower process of exchanging views may help the whole community to reach correct conclusions. In a more precise study, it was later shown that this effect only occurs if learning is difficult and agents have great difficulty distinguishing between better theory \cite{oconnor_rosenstock_2017}. \\
The difficulty of the problem in this model is based on its three parameters:
\begin{itemize}
    \item Payoffs of the two possible actions is very similar.
    \item The community size is small.
    \item The number of evidence collected by individuals in each iteration is small.
\end{itemize} \par

% O'Connor & Weatherall
In the last few years, there have been proposed some interesting extensions to the model proposed by O'Connor and Weatherall, that make it even better at imitating the actual process of opinion spreading in social networks. \par
% Conformity
Conformity is the preferred choice to behave the way the rest of the members of a community do. This drive for conformity is deeply ingrained in human psychology. When conformism is added to the model, cliques of agents who insist on erroneous beliefs emerge. The phenomenon of conformity emergence makes it more difficult for the community to reach a correct consensus \cite{oconnor_weatherall_2018}. \par
% Polarization
If to this model we add social capital of trust or a tendency to conformism we find that given groups are no longer able to reach consensus. When each individual trusts the evidence of those who share his or her views, then extreme dissenting camps form in which everyone listens only to their own. When each individual tries to conform his behavior to the rest of his group, then good ideas are unable to spread between groups.
Social trust matters when an individual treats some sources of information with more respect than others. We can observe this phenomenon when agents trust more evidence provided by members of the same group. Polarization is a phenomenon that occurs frequently in societies. Certain closed subgroups hold a given opinion despite the fact that it is not true. The process behind this is based on unequal treatment of evidence presented by individuals with differing views \cite{oconnor_weatherall_2018_3}. \par
% Propaganda
Industry propagandists shape public opinion by selectively sharing only those results that insidiously support worse behavior. This can also mislead audiences when a group of evidence seekers reaches a consensus by coming to the wrong conclusion. Such a strategy used for public disinformation exploits the natural potential for the randomness of scientific results to mislead. Accounting for such behavior requires the introduction of new classes of nodes into the model, with behavior different from the default - scientific. Agents who are propagandists may share false results, or deliberately publish only the part of their results that discredits a view, and hide those that support it. A small fraction of propagandists can completely sabotage a community's learning process \cite{oconnor_weatherall_2020_1}, \cite{oconnor_weatherall_2020_2}. \par
% endogenous epistemic fractionalization
The problem of epistemic fractionalization was also studied using this model. This is the situation in which people who disagree on one issue tend to disagree on others as well. This related mistrust leads to the endogenous emergence of factions of individuals in the community with highly correlated views on a wide variety of topics. This behavior explains the actual tendency of people to accept full packages of views from the groups to which they belong \cite{oconnor_weatherall_2018_2}.


\section{Summary}
The overview of methods used to model the spread of views presented in this chapter provides an overview of the subject and an understanding of the variety of approaches and challenges faced by researchers working in this area. Different approaches presented, focusing on a macroscopic view of the problem and a microscopic focus on the individual. \par
One of the more advanced and recently focused approaches is the network epistemology model, which covers important concepts that are missing from simpler models. The fact that it captures many social phenomena such as conformism, polarization, and the influence of propaganda allows it to be widely applied. Also, the nature of the model itself and the process of working proposed in it are extremely interesting because of its similarity to the psychological and cognitive processes used by humans on a daily basis. It is this model that the later part of the work will focus on, beginning in the third chapter.


% !TEX encoding = UTF-8 Unicode 
% !TEX root = praca.tex

\chapter{Temporal nature of social networks}

\section{Introduction}
In most of the models described in the first chapter, the time dimension is not used. These models operate on a static network structure in which neither the set of nodes nor the configuration of connections between them changes. Such networks are called static networks. Depending on the domain represented by the network model, edges actually mean some relationship between nodes, but this relationship is very rarely permanent. Networks with a dynamic structure such as the Web, transportation and traffic networks, or communication networks are very popular and surround us from all sides. In the social networks that are the focus of this work, relationships, signaled for example by the contact between individuals, can fade over time. For example, after changing jobs, a person usually does not keep in touch with all former co-workers with whom he or she used to interact on a daily basis. \par
Networks whose topology changes over time are called temporal networks. In this chapter, the systematization of the literature on this type of networks is carried out. Special attention is given to social temporal networks. The needs, addressed by adding a temporal dimension to the models, and the selected approaches proposed to address these needs are presented. \par


\section{Issues and Applications}
The research on temporal networks was undertaken because of the important observation that changes in the structure of the network can influence the processes running in it in the same significant way as the shape of the structure itself. Not paying attention to this fact may cause the omission of an important factor from the analysis and prevent a better and deeper understanding of the studied processes.
% Division
Due to topology dynamics, two groups of networks can be distinguished:
\begin{itemize}
    \item Static - nodes never disappear and all edges are constantly maintained.
    \item Dynamic - nodes can appear and disappear, and edges can be removed and recovered.
\end{itemize} \par
% Division
Another criterion by which temporal networks can be divided is their durability, which defines the intensity of changes occurring in the network structure:
\begin{itemize}
    \item Transient - the dynamics in the topology occur only for a moment, after which the network remains static for a longer period of time.
    \item Continuous - changes in topology occur continuously, there is no periodicity.
\end{itemize} \par
% History
Until recently, the temporal dimension was neglected in most studies. Most often, changes occurring over time were aggregated to a static network, even when the available data allowed for a different approach. One reason for this is that the study of static networks is simpler and far less computationally demanding. \par
% Domains
The field of temporal network research is interdisciplinary. Systems fit for modeling with this approach are very common - genetic regulation, brain functioning, or social networks. The construction of temporal networks is also particularly easy and fitting for electronic communication between individuals who contact each other using, for example, a smartphone and a messaging application. \par
% Applications
Temporal networks are good for modeling when the system being modeled is both random to some degree and contains some patterns and regularities. Modeling using this type of networks is unnecessary when the dynamics of the modeled system is much faster than the dynamics of the contacts occurring in it. 
These approaches move information about when things happen from the dynamical system to the network, the underlying structure on which the dynamics happen. Systems suitable to be modeled as temporal networks are everywhere. A particularly good example of communication, that fits here, is communication between people in social networks, these days often carried out using social media and messaging applications. These are extremely useful data, among other things, in the context of research on the dynamics of information diffusion. For example, messenger applications texts, or SMS messages can take the form of a list of individual interactions.
Temporal networks have been used, among other things, for research on controlling the spread of epidemics. Using them can improve the process and effectiveness of vaccination programs. \par
% Event Sequence
Temporal networks can be built similarly to static networks, using mathematical models or based on real data. One possible representation of this data is an event sequence. This format represents the series of events that occurred between nodes. A single event contains the identifiers of the nodes interacting and the timestamp of the interaction. It is the most granular form of information about interactions between individuals.
The event sequence format can be viewed as raw data, from which a network representation can then be created. \par


\section{Models}
To date, several network models have been proposed to generate synthetic temporal networks. Such networks can serve as references and are also helpful when empirical data are not easily available \cite{holme_saramaki_2012}. Two possible approaches to this problem are described below.

\subsection{Temporal exponential random graphs}
The class of temporal exponential random graphs (TERGMs) models is based on statistical modeling of temporal social networks. It is an extension built on top of exponential random graphs (ERGMs) \cite{kolar_2008}. This extension is achieved by adding additional parameters to the ERGMs specification, which reflect the possible ways in which previous realizations of the network determine its current characteristics. This class of models allows the generation of temporal networks with specific parameters and can account for temporal dependencies in one or more networks.
This type of model is used as a reference model and a way to fine-tune real temporal networks.

\subsection{Models of social group dynamics}
Another group of temporal network models is models of social group dynamics. These models take into account the seriality of social interactions when forming social ties. They focus on maintaining strong connections within subgroups of a network, following the principle that the more often an individual interacts with others in such a group, the less likely he is to leave it  \cite{zhao_2011}. The method generates reliable synthetic temporal networks from real-world data, which can then be tuned as needed.


\section{Representations}
% Intro
Based on models or raw data, a temporal network representation is still needed to facilitate data interpretation and processing, and analysis. There are several possible ways to perform this transformation. At one extreme is the contact sequence method \cite{holme_saramaki_2012}, which transforms an event sequence into a social network in the simplest way. On the other side, it is possible to aggregate an event sequence into a static network. Between these approaches, it is possible to create a temporal social network that consists of a sequence of static networks \cite{brodka_2011}. An example of a social temporal network can be seen in Figure~\ref{rys:tsn} and visualization of different approaches to temporal social network representation are shown in Figure~\ref{rys:representations}.

\begin{figure}
\centering\includegraphics[width=.6\textwidth]{img/rozdzial2/social-temporal-net.png}
\caption{An example of TSN (temporal social network) with five static networks in sequence. (Source: \cite{brodka_2012})}  \label{rys:tsn}
\end{figure}

\subsection{Interaction Sequence}
The simplest and most granular approach is to treat the event sequence with as little interference as possible. This projection ensures equivalence between the network and the event sequence and guarantees that the order of events is preserved. With this representation, each edge in the network exists only at the given time that the interaction in the event sequence relating to it occurred. 

\subsection{Static Network}
Another extreme approach is to create a static network from an event sequence. All interactions are aggregated and the time dimension disappears. Regardless of when an interaction took place, the edge representing it will always be in the network.

\subsection{Sliding Windows}
A compromise solution between the previous two is sliding windows. This is a representation that uses the temporal social network format. It consists of a number of temporally sorted static networks. It is possible to use different approaches to create these sliding windows:
\begin{itemize}
    \item Non-overlapping consecutive windows - single interaction from the event sequence is located in only one window.
    \item Partially overlapping windows - allows you to soften the decision to create a boundary between windows at a given location, but by doing so, certain interactions may be double-counted.
    \item Non-consecutive windows - omits certain time periods from the event time, not creating a window for them, so that some of the interactions will not be taken into account.
    \item Hybrid approaches - a mix of other approaches, there are no rules about windows length and overlapping.
\end{itemize}

\subsection{Incremental Network}
This approach generates a static network based on all interactions that have taken place up to a certain point in time. Each subsequent static network incorporates the full information from the previous one. The network grows over time - new nodes and edges may appear, but once introduced they never disappear again.

\subsection{Recent Events}
The recent events approach creates partially overlapping networks based on time windows whose duration depends on the occurrence time of a given number of recent interactions.

\subsection{CogSNet}
% Intro
CogSNet (cognition-driven social network model) is a novel approach to creating temporal network representations \cite{michalski_2021}. It is the most versatile approach that, given an appropriate set of input parameters, allows the creation of any of the above representations. \par
% Cognition basics
The idea behind this method concerns the use of a more complex algorithm for aggregating interactions between nodes than has been done so far. The scheme of human memory operation is used here. This cognitive perspective allows for better capture of human interactions.
% Machanismfriendship
The level of friendship between individuals is treated, as in the case of human friendship, as a value that is constantly changing over time, rather than being constant between interactions, or changing discretely. The system may work that way due to the use of a forgetting mechanism. This is a function that describes the chance of individuals to recall events. The level of friendship between individuals is determined by examining their interactions with each other. Each interaction bumps up the friendship level by a predefined value, which is one of the model parameters. On the other hand, the friendship level is decreased over time according to the forgetting function. If it is greater than a predefined threshold at any given time, the model decides that there is an active connection between these nodes.
The conducted research confirmed that, with the selection of appropriate parameters, the temporal network created by the CogSNet approach is performing better that the other methods \cite{michalski_2021}. \par

\begin{figure}
\centering\includegraphics[width=.6\textwidth]{img/rozdzial2/representations.png}
\caption{Visualization of the different representation approaches for an example temporal social network with 4 nodes. (Source: \cite{michalski_2021})}  \label{rys:representations}
\end{figure}


\section{Summary}
This chapter has outlined the topic of social temporal networks. Models and popular approaches to representing such networks have been presented.  \par
Conducting research on this type of network is more difficult than with static networks. The number of available methods is also smaller. Nevertheless, the introduction of a temporal factor may be worth the effort because it has a direct impact on the dynamics of processes occurring in social networks. \par
After analyzing the available literature, it was decided to use temporal social networks, created using the CogSNet method, in the later part of this work. The decision was based on the fact that this approach has shown very high potential and great results when tested on real social network data. \par


\chapter{Network Epistemology Model for Temporal Networks}

\section{Introduction}


\section{Related Work}


\section{Basic Network Epistemology Model}


\section{Enhancement for Temporal Network Structure}


\section{Summary}



\chapter{Experimental Setting}

\section{Introduction}
This section will describe the datasets on which the network epistemology model was tested. The configuration of the CogSNet method used to create the temporal social network from the empirical dataset is demonstrated. Then, a description of the research methodology and the model metrics that will be observed in the research is presented.


\section{Network Topologies}
Network epistemology model experiments were conducted on both real social temporal network and static synthetic networks. Some of these model topologies have already been studied in previous publications on learning in social networks, and others have been included because of their distinctive features. These synthetic networks are intended to serve as a reference for the temporal model. \par

\subsection{Temporal Social Network from Empirical Dataset}
In order to investigate the functioning of the temporal network epistemology model, a temporal network was needed. The approaches listed in the second chapter to create such a network come down to two ways - generate it artificially, or use real data. It was decided to use the second way and utilize the NetSense dataset. \par

NetSense is an empirical dataset introduced in \cite{netsense}, that was generated by human interactions, making it an ideal source of raw data to create a temporal social network based on it. The data was collected among a group of students, using a special application installed on their smartphones. The scope of the data covers three years, that is 6 semesters of studies of the group of students. The dataset contains 7,575,865 events, of which 7,096,844 (93.7\%) are text messages and the rest are phone calls. Events were recorded for every student's communication, including those with non-participants. \par
% Cleaning
Before processing the event sequence from the NetSence collection, a series of data cleaning operations were undertaken. Due to the large size of the data, and in order to be as consistent as possible, it was decided to limit only to text messages sent between the students themselves. Duplicates that were sometimes logged by the data collection program were also removed. These operations limited the number of events to 537,575.  \par

The temporal social network was created based on the prepared event sequence data. The network was generated using the CogSNet method described in the second chapter. \par
% Configuration
The configuration of the CogSNet method used was taken from \cite{michalski_2021}, where it was tested using the NetSense collection. The parameters used were those for which, in that publication, the best results were achieved when compared to the surveys completed by the participants forming the NetSense dataset. The configuration is shown in Table~\ref{tab:cogsnet_params}. One day was taken as the resolution of the generated temporal network. This means that 1,103 static networks were generated, one for each day. \par

\begin{table}
    \centering
    \caption{CogSNet parameters}
    \label{tab:cogsnet_params}
    \begin{tabular}{|l|r|}
    \hline
    \textbf{Parameter}  & \multicolumn{1}{l|}{\textbf{Value}} \\ \hline
    Forgetting function & \multicolumn{1}{l|}{exponential}    \\ \hline
    Trace lifetime      & 3 days                              \\ \hline
    Mu                  & 0.3                                 \\ \hline
    Theta               & 0.2                                 \\ \hline
    Lambda              & 0.00563145983483561                 \\ \hline
    Unit                & \multicolumn{1}{l|}{1 hour}         \\ \hline
    \end{tabular}%
\end{table}

\begin{sidewaysfigure}
    \centering\includegraphics[width=\textwidth]{img/rozdzial4/netsense.pdf}
    \caption{Attributes of temporal social network generated from NetSense dataset, using CogSNet method.} 
    \label{fig:netsense}
\end{sidewaysfigure}

% Visualization & stats
After generating the temporal social network, the resulting network was analyzed. Figure~\ref{fig:netsense} shows some seasonality in the activity of the participants, characterized by reduced communication during holiday breaks and inter-semester breaks. In these periods the network is unstable, first a large number of nodes and connections disappear, and then reappear some time later. In order to avoid the impact of such high instability on the study, it was decided to limit the experiments performed on the network to the period of the first semester, between 15 and 125 days, highlighted in Figure~\ref{fig:netsense}. During this time, the number of active nodes and the number of edges are fluctuating, but the changes are not drastic, as in the case of the inter-semester breaks mentioned earlier. During the selected period, there were on average 148 nodes and 157 edges in the network. Example static networks from three different time windows are shown in Figure~\ref{fig:windows}. \par

\subsection{Static Synthetic Networks}
In previous studies of learning in social networks, the network epistemology model has been tested mainly for relatively small (a few to a few dozen nodes) static topologies. \\
The following synthetic networks were selected for the research conducted in this work:
\begin{itemize}
    \item Complete - a network in which each node is connected to everyone else.
    \item Cycle - a network, each node is connected to two others to form a closed chain.
    \item Circle - this graph is an extended version of the cycle, it has one extra node that is connected to all the others.
    \item Erdős–Rényi - a random graph model for creating a network with a given number of nodes, in which each link between any two nodes has an equal probability of occurrence, given as a model parameter.
    \item Watts-Strogatz - a Small-World random network model, allowing to simulate a properties often found in real networks, which are short average path length and the presence of clusters.
    \item Barabási–Albert - a Scale-Free random graph model, which allows the generation of networks using the prefferential attchemnt method, so that the produced network includes hubs and the average node degree follows the power law distribution, a feature observed in many real networks.
\end{itemize} 
The first three structures were studied in \cite{zollman_2007} , while the next ones were added because of their closer similarity to real social networks. When generating these networks for study, it was decided to create them in a size similar to the dynamic structure under study, i.e., for 148 nodes, which number is equal to the average number of active nodes in the temporal network during the period under study. Also, other parameters, important in the case of random graphs, were determined so that the final result resembles the characteristics of the temporal network. For the Erdős-Rényi model, the edge probability parameter was set to 0.0144, which allows generating graphs with an expected edge value of 157, which is the average number of edges of the temporal network over the study period. For the Watts-Strogatz model, the parameter of the base number of neighbors was set to 2, and the probability of switching the edge was set to 0.5. The parameter of the Barabási-Albert model denoting the number of edges created with each new node was set to 2. Examples of these six structures can be seen in Figure~\ref{fig:static_top}.
In this research it was decided to use these structures as references for the temporal model. \par 

\begin{figure}
    \centering\includegraphics[width=\textwidth]{img/rozdzial4/windows.pdf}
    \caption{Exemplary time windows of temporal social network created from NetSense dataset.} 
    \label{fig:windows}
\end{figure}

\begin{figure}
    \centering\includegraphics[width=\textwidth]{img/rozdzial4/static_top.pdf}
    \caption{Example of six different static topologies, generated for given size of 10 nodes.} 
    \label{fig:static_top}
\end{figure}


\section{Methodology}
In all conducted studies, various community characteristics from the temporal network epistemology model were measured. In each simulation, the network state, the average credence value in the community, and the number of individuals voting for each action were collected and recorded. All of these metrics except network state were collected every iteration, and network state only at the end of the simulation. Due to the fact that in the studied temporal network, there is more than one connected component in most time windows, it was decided to record the same data for the largest connected component as well. \\
The simulation can end with one of the highlighted network states:
\begin{itemize}
    \item Correct consensus - in which the credence level of all nodes exceeded 0.99.
    \item Correct disagreement - in which a majority of nodes choose the Beta action, but there is no consensus.
    \item Incorrect disagreement - in which a majority of nodes choose the Alpha action, but there is no consensus.
    \item Incorrect consensus - in which the credence level of all nodes fell below 0.5.
\end{itemize}  
Simulations for static topologies end when one of the consensus options is reached or after 10,000 iterations. For dynamic networks when consensus is reached, or when the last time window is reached. \par

The temporal network epistemology model has several input parameters by which the model can be controlled. The network structures used for experiments have been already discussed above. The sizes of these communities have been set to mimic the temporal network characteristic and will not change in any experiment. The learning problem remained to be specified. \\
The following parameters are required to specify a learning problem:
\begin{itemize}
    \item \textbf{\textit{b}} - the payoff of Beta action.
    \item \textbf{\textit{n}} - the number of trials performed in each individual experiment.
    \item \textbf{\textit{i}} - the number of iterations (experimenting and updating) performed per one time window.
    \item \textbf{\textit{t}} - the threshold of correct consensus acceptance.
\end{itemize}
These values are variable, and their effect on model performance has been tested in experiments. There is no need to specify the payoff of Alpha action, it is set to 0.5. \par
In each experiment, fixed parameters with given values and a variable are specified with their ranges. A suitable number of simulations are run for this configuration so that the decisive influence of the randomness of the underlying model on the results obtained can be eliminated. For the first two experiments, 1,000 simulations were run for each parameter configuration, and 10,000 for the third study. \par
It is important to note that controlling the parameters of the learning problem allows us to control its difficulty. Number trials \textit{n} affects the probability of generating spurious results during the agent's experiments. A larger value of \textit{n} reduces the number of such results that discredit the Beta action, despite its actual superiority. In previous research on this model, \cite{oconnor_rosenstock_2017} has been shown that the value of this parameter significantly affects the model's ability to reach correct consensus. The value of the consensus threshold \textit{t} was set to 0.99 like in \cite{oconnor_weatherall_2018_3}. The number of iterations performed per time window \textit{i} is one of the variable parameters. Network size also affects the ease of learning. Larger communities are more likely to converge to correct consensus, as shown in \cite{zollman_2011}. Keeping the values for these two parameters fixed, the difficulty of the problem is only affected by the value of Beta payoff \textit{b}. In a temporal network, an additional factor that affects the final results is also the learning time, which determines the total number of iterations performed by individuals. \par
The implementation of the prepared model has been made public. The code of the solution and the experiments allowing to reproduce the results generated in this thesis is available on the GitHub platform\footnote{\href{https://github.com/KE4T5/network-epistemology}{https://github.com/KE4T5/network-epistemology}}. \par


\section{Summary}
This section presented the networks on which the experiments were performed. The parameters of the temporal network epistemology model and the attributes that were measured in the experiments were also described. The results of these experiments, in which the effect of using the temporal structure of the network was investigated by manipulating the values of various parameters, are presented in the next section. \par


\chapter{Results and Discussion}

\section{Introduction}
This chapter is devoted to the description of the research conducted and the presentation of the results. Conclusions from the results obtained in the simulations and the impact of these observations on the knowledge of social learning in networks are presented. Finally, potential directions for future work are discussed. \par


\section{Results}

%Problem difficulty for static networks
% Experiment description
As the first study conducted, the effectiveness of reference static networks was compared with a temporal network in solving problems of various difficulties. The difficulty of the problem was controlled by changing the value of the Beta action payoff parameter. Other parameters of the model were kept at fixed, predefined values. Additionally, in order to better verify the effect of the dynamic structure on the temporal network performance, an alternative version of the temporal social network was prepared in which the structure was kept frozen for the last 10 time windows. That is, in the last 10 days, there were no more changes in the connections between individuals and no nodes appeared or disappeared. \par
The Beta action payoff values in the range {0.5001, 0.50025, 0.5005, 0.501, 0.5025, 0.505, 0.51, 0.525, 0.55} were tested. The maximum number of iterations for the static topologies was set to 10,000. To ensure a similar maximum number of possible iterations between the static models and the temporal network, the number of iterations per time window for the temporal network was set to 91. This configuration resulted in the maximum number of iterations for the temporal network being set to 10,010, 10 more than for the static networks. Trials number \textit{n} was equal to 10. 1,000 simulations were run for each parameter configuration. \par

\begin{figure}
    \centering\includegraphics[width=\textwidth]{img/rozdzial5/exp1.pdf}
    \caption{Community performance for various difficulty of the given problem.} 
    \label{fig:exp1}
\end{figure}

\begin{figure}
    \centering\includegraphics[width=\textwidth]{img/rozdzial5/temporal_sim.pdf}
    \caption{Proportion of outcomes for temporal network simulations.} 
    \label{fig:temporal_sim}
\end{figure}

% Observations
The obtained results are shown in Figure~ref{fig:exp1}. It can be observed that all networks except Random and Small-World for Beta action payoff equal to 0.501 reached correct consensus in all simulations performed. For the two topologies mentioned above, this did not happen because some of the generated structures had smaller connected components, containing only two nodes, which may have already received a drawn credence unsupportive of Beta action or fallen into the trap of spurious data at the beginning of the simulation, and once the credence of all individuals in a connected component falls below 0.5, no one makes an attempt to experiment with Beta action and such connected component reaches incorrect consensus. Apart from the random and  Small-World graphs, for Beta action payoff values greater than 0.005, the probability of reaching correct consensus is lowest for both the entire temporal network and its largest connected component. Most significantly, however, the community in the temporal network configuration is not only able to reach consensus, although with a lower probability than in static networks, but for relatively simple problems all simulations ended up reaching the correct consensus on the action payoff. \par
If we look at the average community credence, for simulations with beta action payoff equal to 0.505, we can see that it is comparable for all configurations. The lower likelihood of correct consensus for the temporal network may be due to the fact that certain small connected components, separated from the rest of the individuals fell into incorrect consensus, after which they had no opportunity to contact other nodes since then, and thus no opportunity to change their minds. \par
It is also worth noting that the average time to converge to consensus, regardless of the difficulty of the problem, was always the longest for the temporal network. Only this structure needed a relatively large number of nearly 1,500 iterations to converge to consensus, even for the easiest problem, where other networks were able to handle it very quickly. \par
The version of the temporal network with frozen time windows was tested, but omitted from the visualizations because its performance overlapped almost completely with the regular version of the temporal network. \par
Figure~\ref{fig:temporal_sim} shows the state distribution that the temporal network had at the end of each simulation for different levels of problem difficulty. An interesting observation is that just as in the static networks, in the dynamic network, no simulation ended up converging to the incorrect consensus. \par


% Learning in Temporal Networks
% Experiment description
The research presented in the first chapter shows that network structure is an important factor in the process of opinion formation in social networks. This experiment examined how dynamic structure affects the time course characteristics of this process. Communities organized in reference network structures, and in a dynamic network were given a relatively simple problem to solve. In this case, the goal is not so much to see how successfully each community will perform, but the process itself. To ensure a fair level of conditions, all communities performed an identical number of iterations, that is, 110. In the dynamic network, one iteration was performed for one time window - it is as if individuals performed experiments and contacted each other to update their views exactly once a day for 110 days. In this study, 10,000 simulations were run for each network. Trials number \{textit{n} was equal to 10. \par

\begin{figure}
    \centering\includegraphics[width=\textwidth]{img/rozdzial5/exp2.pdf}
    \caption{The course of community learning process.} 
    \label{fig:exp2}
\end{figure}

% Observations
Averaged over all simulations, the values of the average credence and the number of Beta action voters are shown in Figure~ref{fig:exp2}. A noticeable characteristic of the process flow for the temporal network and its largest connected component is their instability and fluctuating values. The average values of both measured characteristics, for the static networks, seem to be non-decreasing over time, in the case of credence converging to 1 and in the case of beta voter fraction to 100 percent. Instability apparent only in the temporal network, even despite averaging the results of a large number of simulations. This observation indicates that the dynamic structure of the network causes fluctuations in learning. These fluctuations may be a reflection of new nodes joining the network or reactivating those that fell out of the structure some time earlier and were inactive for some time. The chance that such nodes have a low level of credence and choose the Alpha action is greater than that they are followers of the Beta action, especially if they were previously elements of smaller coherent components which are easier to fall into a state of incorrect consensus. \par


\section{Discussion}
Several conclusions can be drawn from the conducted study. First, at least in the case of the temporal social network created on the basis of the NetSense dataset, it can be confirmed both that the community described by this structure is capable of reaching a consensus, and that the dynamic structure of the network has a noticeable influence on both the course and outcome of the learning process. \par
Single cases of convergence to incorrect consensus occurred only in the complete network structure, which is due to the fact that in the rare case when a coincidence leads to the observation of many results discrediting a better action, their large propagation in such a densely connected network has a negative effect on learning, confirming the Zollman effect \cite{zollman_2007}. In other types of static structures and in a dynamic network, no cases of convergence to incorrect consensus were observed, which is probably most influenced by the size of the network. \par
The experiments confirm that the network epistemology model is sensitive to the choice of parameters. A small difference in Beta action payoff can determine the ability of a community to collectively draw correct conclusions and come to a consensus on a topic at a given time. \par
Among the most important observations is the fact that not only the static model structures studied so far, but also the actual social network can collectively solve the problem posed to it in a finite time. One can also observe that for more difficult problems, despite the lack of consensus, almost all or most simulations end with the temporal network prevailing to disperse a better view. For properly configured, easier problems, achieving this result is almost certain. If we look at the largest connected component in the temporal network, we can see that in some cases, even though the whole community may be struggling to solve the problem posed to them, the largest subgroup of the network reaches a correct consensus. This means that those who rejected the better action tend to be on the periphery of the community, gathered in smaller groups supporting the same, but worse, view. The same applies to the time to converge to a consensus. In the full network, it is significantly longer than for static reference networks and for the largest connected component. Convincing initially isolated individuals requires contacting them and cannot be done any other way. Isolated individuals who insist on an erroneous view may and do change their minds, but they must be given evidence to do so. \par
The dynamic structure of the community also has a significant impact on the process of learning itself. The number of people advocating different options can change dynamically and affect the proportion of groups supporting different actions. In no other structure does the average credence or the number of people supporting a better action decrease as the learning process progresses, as it does in a dynamic network. \par
It is important to note, however, that the study was conducted on only a fragment of one temporal network and no strong conclusions or generalizations can be drawn from it, although the results do support the predictions about the effect of dynamized structure on learning. \par


\section{Future Work}
The proposed temporal network epistemology model can be considered as a first step towards studying learning processes in temporal social networks. The research presented here explores only a small fraction of the possibilities for using the model to better understand how human communities collectively work to solve the problems set before them. Several of the possible future directions for this work are outlined below. \par

\subsection{Social Phenomena}
In this work, the proposed temporal network epistemology model is based on the basic learning model proposed in \cite{zollman_2011}. In recent years, there have been several noteworthy modifications of this model proposed in \cite{oconnor_weatherall_2018}, \cite{oconnor_weatherall_2018_2}, \cite{oconnor_weatherall_2018_3}, \cite{oconnor_weatherall_2020_2}, which are further described in the first chapter. These extensions enable to capture some social behaviors like conformity or social trust capital, resulting in social phenomena like polarization and propaganda, which are often observed in real human societies after all. Adding these extensions to the temporal network epistemology model could allow verifying the operation of such phenomena in dynamic networks. \par

\subsection{Experimental Data Expansion}
This study was conducted on only one temporal social network constructed from the NetSense dataset. In order to be able to draw stronger conclusions based on the results presented here, it would be necessary to repeat the experiments on a larger number of temporal networks. There are other reference empirical datasets with different characteristics that could be used for this purpose, such as \cite{chaintreau_2007}, \cite{michalski_2020_2}. Also worth considering is the use of synthetic temporal network models described in the second chapter that allow the generation of such structures. Such a study could confirm or reject the observations presented in this work and shed new light on the problem under study. \par

\subsection{Temporal Structure Synchronization}
From the perspective of temporal network structure, an important direction of development would be to verify the influence of communication and update frequency on the effectiveness and speed of the learning process. A potential experiment could be to test temporal social networks created for time windows of different resolutions. Showing how this process depends on the frequency of information exchange could help to differentiate communities that are better and which are worse prepared to cope with tasks from the category corresponding to the problem class like in the network epistemology model. \par

\subsection{Belief Spread Maximalization}
Another interesting research direction that could be worth exploring is the issue of social influence study. Similar studies have already been conducted for other models of diffusion in social networks \cite{michalski_2014}. More recent research indicates that the relevance of individuals on network spread can be effectively estimated using entropy-based measures \cite{michalski_2020}. A separate direction could also be to study the effect of initial conditions on the learning process. \par


\section{Summary}
This section describes the experiments conducted for different static network configurations and the temporal network for the temporal network epistemology model. The conclusions that can be drawn from these studies are then described. The most relevant observations are that temporal social networks despite their dynamic structure are able to reach a consensus on the network epistemology model and that the dynamic structure affects the learning process stability. In the end, possible directions for the development of this work were presented. \par


% !TEX encoding = UTF-8 Unicode 
% !TEX root = praca.tex

\chapter{Conclusion}
% Intro
The social aspect of learning is a unique and important feature of the human species. The processes behind it are clarified by social learning theory and social epistemology. So far, social learning has been mainly studied using mathematical simulation models for static and synthetic communities. However, it is known that the real human communities have a unique and dynamic character that is difficult to capture by such models. \par
The presented thesis shows how the dynamic structure of a real community can influence the social learning process. It is important because the temporality of the structure of human communities is their immanent feature, which directly affects the processes taking place in them. Taking into account the temporal dimension of community evolution allows for a better and more complete representation of the real dynamics of group opinion formation. \par
% Literature, concept
The work includes a literature review of modeling diffusion processes in social networks, in particular social learning processes, and of modeling temporal networks. Selected algorithms from both fields are described. The concept of the temporal network epistemology model, the author's modification of the network epistemology model, is introduced. The modified model allows simulating the social learning process for temporal networks. Using the model, a study of the effect of temporal social network structure on this process was conducted. The network constructed using the CogSNet method based on the empirical NetSense dataset was used for this purpose. \par
% Conclusions
This study confirmed that the temporality of the social network has a significant impact on the course and outcome of the social learning process. The probability of obtaining correct consensus among temporal social network individuals is lower than in static models, and the process of opinion formation is slower. The temporal social network is also the only structure for which the average course of the process is not stably increasing, but fluctuations and oscillations occur, reflecting changes happening in the community. \par


\section{Future Work}
The proposed temporal network epistemology model can be considered as a first step towards studying learning processes in temporal social networks. The research presented here explores only a small fraction of the possibilities for using the model to better understand how human communities collectively work to solve the problems set before them. Several of the possible future directions for this work are outlined below. \par

\subsection{Social Phenomena}
In this work, the proposed temporal network epistemology model is based on the basic learning model proposed in \cite{zollman_2011}. In recent years, there have been several noteworthy modifications of this model proposed in \cite{oconnor_weatherall_2018}, \cite{oconnor_weatherall_2018_2}, \cite{oconnor_weatherall_2018_3}, \cite{oconnor_weatherall_2020_2}, which are further described in the first chapter. These extensions enable to capture some social behaviors like conformity or social trust capital, resulting in social phenomena like polarization and propaganda, which are often observed in real human societies after all. Adding these extensions to the temporal network epistemology model could allow verifying the operation of such phenomena in dynamic networks. \par

\subsection{Experimental Data Expansion}
This study was conducted on only one temporal social network constructed from the NetSense dataset. In order to be able to draw stronger conclusions based on the results presented here, it would be necessary to repeat the experiments on a larger number of temporal networks. There are other reference empirical datasets with different characteristics that could be used for this purpose, such as \cite{chaintreau_2007}, \cite{michalski_2020_2}. Also worth considering is the use of synthetic temporal network models described in the second chapter that allow the generation of such structures. Such a study could confirm or reject the observations presented in this work and shed new light on the problem under study. \par

\subsection{Temporal Structure Synchronization}
From the perspective of temporal network structure, an important direction of development would be to verify the influence of communication and update frequency on the effectiveness and speed of the learning process. A potential experiment could be to test temporal social networks created for time windows of different resolutions. Showing how this process depends on the frequency of information exchange could help to differentiate communities that are better and which are worse prepared to cope with tasks from the category corresponding to the problem class like in the network epistemology model. \par

\subsection{Belief Spread Maximalization}
Another interesting research direction that could be worth exploring is the issue of social influence study. Similar studies have already been conducted for other models of diffusion in social networks \cite{michalski_2014}. More recent research indicates that the relevance of individuals on network spread can be effectively estimated using entropy-based measures \cite{michalski_2020}. A separate direction could also be to study the effect of initial conditions on the learning process. \par


\section{Summary}
As a conclusion, the thesis was successful in achieving the previously stated goal, and the research findings presented in this paper indicate the clear influence of temporal network structure on social network learning. Most people do not hold blindly to their beliefs, changing them under the influence of evidence presented and interactions with colleagues and friends. These very friendships, like the beliefs themselves, are not permanent. Social ties dynamically evolve over time and thus determine the process of social exchange of knowledge. This aspect should not be ignored when evaluating the potential of social networks in the context of solving epistemic problems. \par


% Bibliografia
% W spisie pojawią się tylko pozycje cytowane w tekście, np.: \cite{aizawa_groundwater_2009}.
\bibliography{literatura}
\bibliographystyle{dyplom}


% Spisy rysunków listingów i tabel 
% Można włączyć gdyby opiekun pracy sobie życzył :)
\listoffigures
%\listof{listing}{Spis listingów}
\listoftables

% Dodatki - tu można umieścić duże objętościowo materiały 
% - Projekt interfejsu użytkownika, 
% - Scenariusze wszystkich przypadków użycia

%\appendixpage
%\appendix
%\input{dodatek}
%\addappheadtotoc



\end{document}
