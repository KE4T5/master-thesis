% !TEX encoding = UTF-8 Unicode 
% !TEX root = praca.tex

\chapter*{Introduction}

% Wprowadzenie w tematykę

% 1. Społeczny aspekt komunikacji wśród ludzi
Humans are extremely social animals. The specific abilities that our species has developed in the course of the cognitive revolution have allowed humans to dominate other organisms on our planet[Harari, Yuval Noah. 2015. Sapiens: A Brief History of Humankind.]. One crucial element in this process is the incredibly powerful and effective, evolved ability of homo sapiens to communicate and exchange large amounts of information about social relationships and the world around us. Given the evolutionary history of our species, it should come as no surprise that human systems for forming views and opinions are also social in nature. Humans pass information from one person to another, and as a result a large number of our views are dependent on the environment of our contacts. It is this evolutionarily evolved and inherent human way of spreading knowledge that is the foundation of today's extremely complex and advanced culture and science.

% 2. Modelowanie
Due to the above-mentioned role of the social aspect of information exchange in the social influence process, researchers have long been carrying out research that helps to understand the characteristics of this process and its complexity. There is a growing interest in modelling in the scientific community. What is the best community structure for spreading information? What conditions must be met to allow the community to reach consensus on a given topic? What influences the speed of the spread of information and opinions? Various models developed by researchers in recent years provide answers to these questions. Through the use of mathematical modelling it is possible, to a certain extent, to predict spreading scenarios and to gain insight into the complex dynamics of the social influence process. This research is extremely important and useful, but often the experiments are conducted on model network structures. Therefore, the results of such research should always be approached with caution and, if possible, verified.

% 3. Network Epistemology Model
Many of these models are based on the functioning of social networks, that have been identified as a major factor in determining how societies move towards adopting ideologies. One social network model that provides an interesting way of capturing aspects of social learning in human societies and its impact on theory adoption is the Network Epistemology Model. It is an agent-based model that offers an imitation of the human and especially scientific way of analysing information and updating beliefs. It has so far mainly been used to model the spread of knowledge in scientific communities, but it can also be applied in a more general context, to study any community of people receiving evidence and sharing their beliefs.

% 4. Temporalna natura sieci społecznych
An important feature of the social networks studied by the researchers is their temporality - acquaintances and friendships appear and disappear, old relationships are forgotten, but new people are met. Most real networks are temporal and their structure - the confluence of nodes and edges - is changing over time. The temporal structure of link activity can influence the dynamics of interaction in a network just like its topology. Researchers have already proposed many models, which attempt to capture this aspect of social networks in different ways. The novel cognition-driven social network (CogSNet) model proposes unique approaches based on social perception.\\

% Co zawiera praca + odeniesienie do tytułu
This work addresses the topic of modelling social influence in temporal social networks. The question is whether the models that are currently used in the domain correctly represent the dynamics of opinion propagation in real social networks. Special attention is given to the Network Epistemology Model, for which so far no extension has been developed to make the model work for temporal networks, and to the CogSNet temporal network model.

The dynamics of the social network mentioned in the title of the paper is understood here as the temporality of the network structure, and the dynamics of the spread of information as the process of social influence - the spread of views and theories in communities.



\section*{Objective}

The main aim of the thesis was to investigate the influence of the temporal nature of real-world networks on the process of information diffusion and, in particular, process of learning and social influence in social networks.

The main purpose of this paper is to analyse how the results of one of the typical models of spread of influence <model> differ depending on the strategy of building the social network used later for ...


\section*{Scope of work}
% TODO: name of telco dataset
The work involved developing a concept and implementing a modification of the Network Epistemology Model, in a way that allows the model to work for temporal networks. Implementation of the basic model and the extension was made in Python programming language. The structure of the real temporal network was generated using the CogSNet model and a set of interactions between students using instant messaging applications on smartphones from TelcoData dataset. Using the developed model and the generated temporal network structure, a series of studies were performed including an analysis of the influence of the temporal structure of the network on the consensus formation process and its characteristics.\\

The paper is divided into five chapters. The first and second chapter presents a review of the literature on social influence modelling in temporal social networks. In the third chapter, the Network Epistemology Model is presented in detail, as well as the author's concept of modifying this model to make it work in temporal networks. The fourth chapter focuses on the description of the methodology and research conducted using the model. In the fifth chapter there is an analysis of the obtained results and a discussion about possible further research using the model.

% -------------------------------------------


% Dodatkowe
% Illustrate how the social aspects of science can influence theory adoption and lead to potential problems.
% Celem pracy jest opracowanie modelu Model ma umożliwiać analizę zjawiska narzędzie analizy oraz badanie wpływu różnych czynników na jego zasięg i dynamikę