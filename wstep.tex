% !TEX encoding = UTF-8 Unicode 
% !TEX root = praca.tex

\chapter{Introduction}
% Social aspect
Humans are extremely social animals. The specific abilities that our species has developed in the course of the cognitive revolution have allowed humans to dominate other organisms on our planet. One crucial element in this process is the incredibly powerful and effective, evolved ability of homo sapiens to communicate and exchange large amounts of information. These behaviors have been described by social learning and social choice theory. Given the evolutionary history of our species, it should come as no surprise that human systems for forming views and opinions are also social in nature. Humans pass information from one person to another, and as a result, a large number of our views are dependent on the environment of our contacts. This inherent human way of spreading knowledge is the foundation of today's extremely complex and advanced culture and science. \par
% Modeling
Due to the above-mentioned role of the social aspect of information exchange in the social learning process, researchers have long been carrying out research that helps to understand the characteristics of this process and its complexity. There is a growing interest in modeling in the scientific community. What is the best community structure for spreading information? What conditions must be met to allow the community to reach a consensus on a given topic? What influences the speed of the spread of information and opinions? Various models developed by researchers in recent years provide answers to these questions. Through the use of mathematical modeling, it is possible, to a certain extent, to predict spreading scenarios and to gain insight into the complex dynamics of the social learning process. This research is extremely important and useful, but often the experiments are conducted on static and synthetic network structures. Therefore, the results of such research should always be approached with caution and, if possible, verified. \par
% Network Epistemology Model
Many of these models are based on the network structure, which have been identified as a major factor in determining how societies move towards adopting ideologies. One social network model that provides an interesting way of capturing aspects of social learning in human societies and its impact on theory adoption is the network epistemology model. It is an agent-based model that offers an imitation of a human and especially a scientific way of analyzing information and updating beliefs. It has so far mainly been used to model the spread of knowledge in scientific communities, but it can also be applied in a more general context, to study any community of people receiving evidence and sharing beliefs. \par
% Temporal networks
An important feature of the social networks studied by the researchers is their temporality - acquaintances and friendships appear and disappear, old relationships are forgotten, but new people are met. Most real networks are temporal and their structure - the configuration of nodes and edges - is changing over time. The temporal structure of link activity can influence the dynamics of interaction in a network just like its topology. Researchers have already proposed many models, which attempt to capture this aspect of social networks in different ways. The novel cognition-driven social network (CogSNet) model proposes unique approaches based on social perception. \par
% Content
This work addresses the topic of social learning modeling in temporal social networks. The question is whether the currently used models correctly represent the dynamics of opinion propagation in real social networks. Special attention is given to the network epistemology model, for which so far no extension has been developed to make the model work for temporal networks, and to the CogSNet temporal network representation model. \par
The dynamics of the social network mentioned in the title of the paper is understood here as the temporality of the network structure, and the dynamics of the spread of information as the process of social learning - the spread of views and theories in communities. \par


\section{Objective}
The aim of the thesis was to investigate the influence of the temporal nature of real-world networks on the process of information diffusion and, in particular, the process of social learning networks. This study is accomplished by utilizing an extension of the network epistemology model proposed in this work, which made it possible to analyze the impact of dynamic community structure on problem solving ability and the nature of this process. \par


\section{Scope of Work}
The work involved developing a concept and implementing a modification of the network epistemology model, in a way that allows the model to work for temporal social networks. Implementation of the basic model and the extension was made in Python programming language. The representation of the temporal social network was generated using the CogSNet model and an empirical NetSense dataset. Using the developed model and the generated temporal network structure, a series of studies were performed including an analysis of the influence of the temporal structure of the network on the consensus formation process and its characteristics. \par
This thesis is divided into seven chapters. The first is the introduction. The second and third chapters present a review of the literature on the social learning modeling in temporal social networks. In the fourth chapter, the network epistemology model is presented in detail, as well as the author's concept of modifying this model to make it work for temporal networks. The fifth chapter focuses on the description of the empirical dataset and the methodology of studies. The sixth chapter presents the experiments performed and follows with a discussion on the results. The seventh chapter contains conclusion and description of potential directions for the further development of this work. \par
