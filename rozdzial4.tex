\chapter{Experimental Setting}

\section{Introduction}
This section will describe the datasets on which the network epistemology model was tested. The configuration of the CogSNet method used to create the temporal social network from the empirical dataset is demonstrated. Then, a description of the research methodology and the model metrics that will be observed in the research is presented.


\section{Network Topologies}
Network epistemology model experiments were conducted on both real social temporal network and static synthetic networks. Some of these model topologies have already been studied in previous publications on learning in social networks, and others have been included because of their distinctive features. These synthetic networks are intended to serve as a reference for the temporal model. \par

\subsection{Temporal Social Network from Empirical Dataset}
In order to investigate the functioning of the temporal network epistemology model, a temporal network was needed. The approaches listed in the second chapter to create such a network come down to two ways - generate it artificially, or use real data. It was decided to use the second way and utilize the NetSense dataset. \par

NetSense is an empirical dataset introduced in \cite{netsense}, that was generated by human interactions, making it an ideal source of raw data to create a temporal social network based on it. The data was collected among a group of students, using a special application installed on their smartphones. The scope of the data covers three years, that is 6 semesters of studies of the group of students. The dataset contains 7,575,865 events, of which 7,096,844 (93.7\%) are text messages and the rest are phone calls. Events were recorded for every student's communication, including those with non-participants. \par
% Cleaning
Before processing the event sequence from the NetSence collection, a series of data cleaning operations were undertaken. Due to the large size of the data, and in order to be as consistent as possible, it was decided to limit only to text messages sent between the students themselves. Duplicates that were sometimes logged by the data collection program were also removed. These operations limited the number of events to 537,575.  \par

The temporal social network was created based on the prepared event sequence data. The network was generated using the CogSNet method described in the second chapter. \par
% Configuration
The configuration of the CogSNet method used was taken from \cite{michalski_2021}, where it was tested using the NetSense collection. The parameters used were those for which, in that publication, the best results were achieved when compared to the surveys completed by the participants forming the NetSense dataset. The configuration is shown in Table~\ref{tab:cogsnet_params}. One day was taken as the resolution of the generated temporal network. This means that 1,103 static networks were generated, one for each day. \par

\begin{table}
    \centering
    \caption{CogSNet parameters}
    \label{tab:cogsnet_params}
    \begin{tabular}{|l|r|}
    \hline
    \textbf{Parameter}  & \multicolumn{1}{l|}{\textbf{Value}} \\ \hline
    Forgetting function & \multicolumn{1}{l|}{exponential}    \\ \hline
    Trace lifetime      & 3 days                              \\ \hline
    Mu                  & 0.3                                 \\ \hline
    Theta               & 0.2                                 \\ \hline
    Lambda              & 0.00563145983483561                 \\ \hline
    Unit                & \multicolumn{1}{l|}{1 hour}         \\ \hline
    \end{tabular}%
\end{table}

\begin{sidewaysfigure}
    \centering\includegraphics[width=\textwidth]{img/rozdzial4/netsense.pdf}
    \caption{Attributes of temporal social network generated from NetSense dataset, using CogSNet method.} 
    \label{fig:netsense}
\end{sidewaysfigure}

% Visualization & stats
After generating the temporal social network, the resulting network was analyzed. Figure~\ref{fig:netsense} shows some seasonality in the activity of the participants, characterized by reduced communication during holiday breaks and inter-semester breaks. In these periods the network is unstable, first a large number of nodes and connections disappear, and then reappear some time later. In order to avoid the impact of such high instability on the study, it was decided to limit the experiments performed on the network to the period of the first semester, between 15 and 125 days, highlighted in Figure~\ref{fig:netsense}. During this time, the number of active nodes and the number of edges are fluctuating, but the changes are not drastic, as in the case of the inter-semester breaks mentioned earlier. During the selected period, there were on average 148 nodes and 157 edges in the network. Example static networks from three different time windows are shown in Figure~\ref{fig:windows}. \par

\subsection{Static Synthetic Networks}
In previous studies of learning in social networks, the network epistemology model has been tested mainly for relatively small (a few to a few dozen nodes) static topologies. \\
The following synthetic networks were selected for the research conducted in this work:
\begin{itemize}
    \item Complete - a network in which each node is connected to everyone else.
    \item Cycle - a network, each node is connected to two others to form a closed chain.
    \item Circle - this graph is an extended version of the cycle, it has one extra node that is connected to all the others.
    \item Erdős–Rényi - a random graph model for creating a network with a given number of nodes, in which each link between any two nodes has an equal probability of occurrence, given as a model parameter.
    \item Watts-Strogatz - a Small-World random network model, allowing to simulate a properties often found in real networks, which are short average path length and the presence of clusters.
    \item Barabási–Albert - a Scale-Free random graph model, which allows the generation of networks using the prefferential attchemnt method, so that the produced network includes hubs and the average node degree follows the power law distribution, a feature observed in many real networks.
\end{itemize} 
The first three structures were studied in \cite{zollman_2007} , while the next ones were added because of their closer similarity to real social networks. When generating these networks for study, it was decided to create them in a size similar to the dynamic structure under study, i.e., for 148 nodes, which number is equal to the average number of active nodes in the temporal network during the period under study. Also, other parameters, important in the case of random graphs, were determined so that the final result resembles the characteristics of the temporal network. For the Erdős-Rényi model, the edge probability parameter was set to 0.0144, which allows generating graphs with an expected edge value of 157, which is the average number of edges of the temporal network over the study period. For the Watts-Strogatz model, the parameter of the base number of neighbors was set to 2, and the probability of switching the edge was set to 0.5. The parameter of the Barabási-Albert model denoting the number of edges created with each new node was set to 2. Examples of these six structures can be seen in Figure~\ref{fig:static_top}.
In this research it was decided to use these structures as references for the temporal model. \par 

\begin{figure}
    \centering\includegraphics[width=\textwidth]{img/rozdzial4/windows.pdf}
    \caption{Exemplary time windows of temporal social network created from NetSense dataset.} 
    \label{fig:windows}
\end{figure}

\begin{figure}
    \centering\includegraphics[width=\textwidth]{img/rozdzial4/static_top.pdf}
    \caption{Example of six different static topologies, generated for given size of 10 nodes.} 
    \label{fig:static_top}
\end{figure}


\section{Methodology}
In all conducted studies, various community characteristics from the temporal network epistemology model were measured. In each simulation, the network state, the average credence value in the community, and the number of individuals voting for each action were collected and recorded. All of these metrics except network state were collected every iteration, and network state only at the end of the simulation. Due to the fact that in the studied temporal network, there is more than one connected component in most time windows, it was decided to record the same data for the largest connected component as well. \\
The simulation can end with one of the highlighted network states:
\begin{itemize}
    \item Correct consensus - in which the credence level of all nodes exceeded 0.99.
    \item Correct disagreement - in which a majority of nodes choose the Beta action, but there is no consensus.
    \item Incorrect disagreement - in which a majority of nodes choose the Alpha action, but there is no consensus.
    \item Incorrect consensus - in which the credence level of all nodes fell below 0.5.
\end{itemize}  
Simulations for static topologies end when one of the consensus options is reached or after 10,000 iterations. For dynamic networks when consensus is reached, or when the last time window is reached. \par

The temporal network epistemology model has several input parameters by which the model can be controlled. The network structures used for experiments have been already discussed above. The sizes of these communities have been set to mimic the temporal network characteristic and will not change in any experiment. The learning problem remained to be specified. \\
The following parameters are required to specify a learning problem:
\begin{itemize}
    \item \textbf{\textit{b}} - the payoff of Beta action.
    \item \textbf{\textit{n}} - the number of trials performed in each individual experiment.
    \item \textbf{\textit{i}} - the number of iterations (experimenting and updating) performed per one time window.
    \item \textbf{\textit{t}} - the threshold of correct consensus acceptance.
\end{itemize}
These values are variable, and their effect on model performance has been tested in experiments. There is no need to specify the payoff of Alpha action, it is set to 0.5. \par
In each experiment, fixed parameters with given values and a variable are specified with their ranges. A suitable number of simulations are run for this configuration so that the decisive influence of the randomness of the underlying model on the results obtained can be eliminated. For the first two experiments, 1,000 simulations were run for each parameter configuration, and 10,000 for the third study. \par
It is important to note that controlling the parameters of the learning problem allows us to control its difficulty. Number trials \textit{n} affects the probability of generating spurious results during the agent's experiments. A larger value of \textit{n} reduces the number of such results that discredit the Beta action, despite its actual superiority. In previous research on this model, \cite{oconnor_rosenstock_2017} has been shown that the value of this parameter significantly affects the model's ability to reach correct consensus. The value of the consensus threshold \textit{t} was set to 0.99 like in \cite{oconnor_weatherall_2018_3}. The number of iterations performed per time window \textit{i} is one of the variable parameters. Network size also affects the ease of learning. Larger communities are more likely to converge to correct consensus, as shown in \cite{zollman_2011}. Keeping the values for these two parameters fixed, the difficulty of the problem is only affected by the value of Beta payoff \textit{b}. In a temporal network, an additional factor that affects the final results is also the learning time, which determines the total number of iterations performed by individuals. \par
The implementation of the prepared model has been made public. The code of the solution and the experiments allowing to reproduce the results generated in this thesis is available on the GitHub platform\footnote{\href{https://github.com/KE4T5/network-epistemology}{https://github.com/KE4T5/network-epistemology}}. \par


\section{Summary}
This section presented the networks on which the experiments were performed. The parameters of the temporal network epistemology model and the attributes that were measured in the experiments were also described. The results of these experiments, in which the effect of using the temporal structure of the network was investigated by manipulating the values of various parameters, are presented in the next section. \par
