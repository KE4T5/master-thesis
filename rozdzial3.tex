\chapter{Temporal Network Epistemology Model}

\section{Introduction}
In this chapter, the temporal epistemic network model is presented. First, a list of publications and studies that are related to the diffusion processes in social networks is listed. Then, the basic version of the network epistemology framework, on which the rest of the paper is based, is described. Finally, a novel modification to the model is presented, which dynamizes the so far static network structure of the basic model and makes it operational for temporal social networks.


\section{Related Work}
Temporal networks have already been used many times to model diffusion processes in networks. It is known that adding a time dimension can significantly affect these processes. By using a temporal approach, it is possible to gain a deeper and better understanding of the spreading process. To date, there has not yet been an attempt to analyze social learning using temporal networks, but other diffusion processes in social networks have been studied in this way.  \par
All the studies presented below indicate that the dynamic network structure has a significant impact on various processes related to diffusion in networks. It is likely that the problem of learning in social networks can be similarly affected. \par
% Threshold model of cascades in temporal networks
In this work \cite{karimi_holme_2012}, an extension of the classical threshold model is proposed that takes into account the time dimension of the network. In this model, influence can only be generated by contacts from the recent past. It was observed that the use of a temporal network significantly affected the studied systems. It was also shown that network burstiness can in some cases accelerate the diffusion process. \par
% Impact of Non-Poissonian Activity Patterns on Spreading Processes
Another paper \cite{barabasi_2007} points out that people do not contact each other uniformly in the context of time. The distribution of contact times is highly tailed, which is clearly related to social network burstiness. Incorporating this property into the epidemic model of spread made the extinction time of epidemics significantly longer than in the baseline models. The predictions of the spreading process generated by the new model were a much better fit to the actual computer virus spreading data. \par
% Seed Selection for Spread of Influence in Social Networks: Temporal vs. Static Approach
A study \cite{michalski_2014} was also conducted to show how the use of temporal social networks can affect the efficiency of finding the most influential nodes in a linear threshold model. After verifying the approach on five real data sets, the authors found out that using a dynamic structure for seed selection can significantly increase the number of influenced nodes. \par
% Predicting and controlling infectious disease epidemics using temporal networks [epidemic temporal model]
In another study \cite{holme_2013}, the authors focused on verifying the effect of dynamic network structure on the immunization process. The performed experiments showed that in this case, the temporal structure of the network is as important as the topology itself. \par


\section{Basic Network Epistemology Model}
The basic concept of the network epistemology model presented by Zollman in \cite{zollman_2011} assumes that the model consists of a fixed set of agents that represent people in a social network. The agents are connected to each other by symmetric relationships in a graph structure. \par
These agents are faced with the problem of choosing between two theories with different effectiveness. In this case, the dilemma can be described as a two-armed bandit problem, in which the Alpha arm returns a payoff equal to 1 with a probability of 0.5, and the Beta arm returns the same payoff but with a slightly higher probability equal to 0.5 + epsilon. The agents know the effectiveness of the Alpha arm, but they do not know whether the Beta arm is better or worse. The problem comes down to choosing the better arm. As an example, a group of doctors may use two different therapies with different efficacy to treat a certain disease. One of these treatments has been used for a long time and its efficacy is well known, and the other, an innovative one has similar but unknown efficacy that may be slightly better or worse. \par
Actors choose an action based on their current credence, represented by a real number between 0 and 1. This credence can be interpreted as the probability that the Beta action is better than the Alpha action. If an agent's credence is greater than 0.5, it will choose the Beta action, otherwise the Alpha action. \par
The process of exchanging views in this model begins by drawing agents' confidence from a uniform distribution over the interval of possible credences. Then, in each iteration, agents take two activities: experimenting and updating beliefs. Within the experimentation phase, each agent performs its preferred action a specified number of times and observes the results. The results of an action are drawn from a binomial distribution with an action-specific probability of success. For example, an actor whose confidence is equal to 0.8 has 80\% confidence that the action Beta is better, so he chooses to perform it a certain number of times, some of which will be successful. Credences are then updated based on the collected results. Actors do this using Bayes' rule (strict conditionalization). Actors who performed the Alpha action, regardless of the observed results do not update their credence, because its payoff is well known. The observations are share among the actors. The credence update is performed not only on the basis of the evidence observed by a specific agent but also on the basis of the evidence collected by its neighbors. This means that any agent whose neighbors have performed a Beta action will update their views. \par
This process proceeds iteratively until the entire community converges to one of the beliefs, reaching consensus. This can occur when all agents have sufficiently low credence, less than or equal to 0.5 (incorrect consensus), or sufficiently high credence - above an arbitrary threshold, usually set to 0.99 (correct consensus). In this version of the model, the community usually converges to one of the beliefs in finite time. \par


\section{Modification for Temporal Social Networks}
The network epistemology model described above does not take into account the time factor. It seems, the temporality of the structure may be important in the context of the spread of beliefs, so the concept of such a modification is proposed below. \par
The temporal version is based on the utilization of the temporal social network concept described in the second chapter. A temporal social network consists of a series of static social networks. Here, we will simply assume that a certain number of iterations of the basic model will be carried out for each of these static networks, and then the network structure will be changed for the next one. \par
% Changes
A major change resulting from this approach is that simulations can only be run up to a certain number of iterations, equal to the product of the number of time windows in the temporal social network and the number of iterations runs in each window. Therefore, there may be a situation where no consensus can be reached by that time. \par
% Isolated nodes
This approach also makes it necessary to introduce some modifications regarding the details of the model. In the case of temporal networks, nodes can lose connection with all their neighbors and be isolated from the rest of the network. Such nodes are labeled as inactive. In contrast, active nodes are those that have at least one neighbor. For this reason, inactive nodes are not considered when checking the consensus condition. Such nodes also do not conduct experiments but retain their credence level and when they return to the network in the future, their belief is as it was when they disappeared from the network. \par


\section{Summary}
This chapter presented research indicating that dynamic network structure affects various processes related to diffusion in social networks. Since learning in social networks has so far only been studied for static networks, it was thus decided to introduce a modification of the network epistemology model that allows it to work for temporal social networks. Both the base modal and the concept of the modification were described. The extended model was then used to study the process of belief propagation in social networks. The results of this research are presented in the following sections. \par
