% !TEX encoding = UTF-8 Unicode 
% !TEX root = praca.tex


\chapter{Social Networks and Learning}

\section{Introduction}
This chapter will discuss the topic of learning in social networks. Key concepts will be presented as well as questions that researchers studying this topic are trying to answer. It will then list popular and important models used over the years to study the spread of information and beliefs.


\section{Issues and Applications}
% Social Systems 
Social systems are extremely difficult to describe and explain, despite developments in statistical models and computing power. What we observe in these systems is the effect of the influence of a huge number of internal and external factors. The internal ones include the flow of information between the participants of social life, the lack of knowledge of the initial conditions of the microscopic state of systems, and the precisely defined laws governing them or, more generally, social laws. External factors, in turn, include political events, natural factors (disasters, climatic phenomena), and technological factors, as well as the influence of other societies. Moreover, the individuals, that form the communities, are intelligent beings, who can learn and adapt their behavior to changing circumstances. The enormous number of these factors and their interconnectedness makes human societies some of the most complex systems existing in the Universe. All this together means that a correct and comprehensive description of these communities requires taking into account the information available at different levels of society. The need for such a holistic description is present also for many other complex systems in nature. \par
% Networks and social networks intro
Networks are the data structures that allow to characterize and study many complex systems such as infrastructure grids, ecosystems, climate, but also people societies. A social network is a specific type of complex network whose structure consists of actors (people, organizations) as nodes, and social interactions between the actors as links \cite{green_bossomaier_2000}. Social networks analysis is a vast and interdisciplinary academic field with origins in sociology, social psychology, graph theory, and statistics. It focuses on the study of relationships between individuals, groups of individuals, and even entire societies. \par
% Learning in networks
The formation of opinions and sharing the information between individuals is determined by their social networks. Obtaining information and consulting opinions with other humans is an evolutionary trait, that gives humans a huge advantage over other animals that do not have this ability. People tend to consult their friends and neighbors for opinions over a new firm they want to apply for, discuss the credibility of political parties, or just ask what movie to watch. A good understanding of the way that social networks influence the process of opinion formation and sharing is critical for understanding these phenomenons themselves.
One of the most important sections of social network analysis is the study of information diffusion and learning in social networks. One of the most common ways to disseminate new ideas and opinions is to learn from friends and neighbors. Most social learning processes have two distinct components. First, is the exchange of ideas between people who have a made opinion about something. Second, is the spread of new information to a person unfamiliar with the subject. Social relationships allow to information transmission by observing other people's decisions and by talking and sharing opinions. Understanding how individuals use information from their social environments and the implications of this process is important in many contexts. The key role played by the transmission of this information has been demonstrated for various phenomena such as product selection \cite{trusov_2008}, voting in elections \cite{allen_2002}, and job search \cite{montgomery_1991}. It is well known that the characteristics of contacts configuration, in a form of network structure, influences many aspects of social life. Among other things, the studies conducted so far have proven that even a community of perfectly rational individuals can come to incorrect conclusions. People who alone would have no problem distinguishing truth from falsehood can form irrational groups \cite{zollman_2013}. \par
% Questions
Some of the important questions regarding the formation of the beliefs in social networks are:
\begin{itemize}
    \item Which individuals have the biggest influence over the beliefs in a whole society?
    \item Whether individuals will conform?
    \item Whether individuals in society come to the right belief?
    \item Whether dispersed information is aggregated accurately and efficiently?
    \item How long it takes for the individuals learn?
\end{itemize} \par
% Ending
The social learning process is distributed through space and time. It usually requires the involvement of up to thousands of people. Theories adaptation may take many years and numbers of interactions between the people. This is why observational and experimental methods are not sufficient for studying large-scale social effects. The motivation for simulation and modelling is the complexity of social systems of non-linear, multi-level, and dynamic interactions. Modelling different types of social functions is often not feasible by other computational methods. Computer simulations and modelling are extremely useful in case when it is hard to investigate a community of people empirically. Testing hypotheses, about which factors might make a difference to how such groups of people learn is possible by building simple programs that can mimic groups of people sharing views. The results of such simulations can help to interpret what we experience in the real world - and even suggest new ways of seeing the full complexity of human societies. Analysis of how these models work, confirmed by empirical studies can change our fundamental understanding of ourselves.



\section{Models and Approaches}
% Computer simulations
Support of social sciences by mathematical models as well as statistical and computational methods has been used successfully for a long time. All this is done in order to describe social phenomena, which are among the most complex in nature. In recent years, mathematical models are more and more effectively used to describe phenomena observed in complex systems. These systems, because of their complex mathematical description (significantly related to their nonlinearity, which prevents obtaining accurate results of analytical calculations), can be studied mainly by approximate or numerical methods, observing the time evolution of their states. As a result of the development of computer techniques, many works have been proposed describing systems, whose structure and dynamics have so far been based on quantitative analysis. Mathematical description enables a better understanding of the processes happening in complex networks \cite{holme_saramaki_2012}. An important element of such research, is the cognitive aspect, as complex networks are commonly found in society. The phenomena occurring inside them have, on the one hand, a system-specific character, while on the other hand, in many cases they have a universal character, showing similarity to phenomena occurring in other natural systems. \par
% Historycznie
The first studies of social learning were concentrated on social influence topics and explored the theory of opinion leaders. These focused on the question of which individuals tend to become opinion leaders. Subsequently, scientists have consistently developed more complex and realistic models that have made it possible to answer the wider and wider range of questions presented in the previous section. For years, attempts have been made to create models that reproduce changes in community beliefs over time. Although it has been a problem to relate many theoretical results to real data, the situation has changed in recent years thanks to data obtained from social media networks. So far, a very large number of opinion dynamics models have been developed. Among them, there are some simple fundamental models that allow us to understand the most important phenomena and capture the overall sense of the problem. \par
% Klasyfikacja modeli
Models can be divided into two main, fundamentally different types: macroscopic and microscopic. The first ones help to answer the questions of how? and how much? They do not represent what happens at the microscale, at the level of individual units of analysis. They model the behaviour of appropriate average values. The main disadvantage of such models is the lack of an answer to the question about the causes of occurring phenomena. Microscopic models are created in order to try to answer the question of why something happens. \\
In the case of the economic and social sciences, models of the following type are distinguished:
\begin{itemize}
    \item Microsimulation, where changes in the states of objects are determined by certain deterministic or stochastic rules.
    \item Agent Based Models, in which the system under study is a set of agents that interact with each other according to certain model-dependent rules.
\end{itemize} \par
% Agent Based Modelling
A frequently used approach is agent based modelling, in which a single agent is equivalent to a person and the whole system consists of a network of such agents. An agent, as the basic element of a system, has certain numerically described characteristics like opinion and usually interacts with other agents or external factors.  Both the characteristics of a single agent and the rules of interaction are dependent on the specific model. An opinion is a numerical representation of an agent's belief, in most cases, it can be an integer, a continuous number, or a vector. \\
The general scheme in agent based models can be presented as follows:
\begin{enumerate}
    \item Single agents or groups of agents are arranged - the algorithm for selecting agents for meetings depends on the model.
    \item A discussion follows, as a result of which agents can change their opinion - here again, the details of the interaction depend on the specific model.
    \item The procedure is repeated.
\end{enumerate} \par
% Bayesian or myopic
Other criteria by which the different approaches can be divided are whether learning is Bayesian or myopic and whether individuals learn from specific signals or by observing actions taken by others. The advantage of Bayesian updating is its solid normative basis but it's often complex. The observation and updating of views in this model are done in a sophisticated way. Alternatively, simpler models may not pay attention to the actions of specific individuals, only observe their opinion and update their own by averaging. \\
The following subsections will discuss several types of different models that have so far been frequently used to study the problem of social learning.


\subsection{Epidemic Models}
Epidemic models have been used mainly to predict the spread of diseases and the development of epidemics. However, it has also been used to model the diffusion of information. In these models, concepts or views are treated as viruses that spread between individuals within a social network. There are different ways in which such an infection can spread. Some models assume that any individual who neighbours an infected one will also become infected themselves. In others, however, ideas spread whenever a certain percentage of neighbours become infected. \par
The most popular approach to epidemic modelling is Susceptible-Infectious-Recovered (SIR), in which each individual can be in one of 3 states. Many modifications of this model have been developed that take into account births and deaths, lack of immunity after recovery (SIS) or temporal immunity (SIRS) \cite{bailey_1975}. Some studies have shown that in some specific situations epidemic models can correctly model the spread of views and ideas.
These models have been used successfully to study the propagation dynamics of topics \cite{guha_2004}. 


\subsection{Sznajd Model}
The concept of the Sznajd model \cite{sznajd_2001} is based on a fact known in social psychology as social proof. If a person with an opinion that differs from ours tries to convince us that he is right, he has a difficult task ahead of him. But when, at the same time, two or more people agree with each other, they can succeed incomparably easier. Based on this insight, a model of opinion dynamics was developed and proved to be a huge success. It is a simple but yet important variation of the prototypical Ising system. It is still studied today and continues to provide many interesting results. \par
In the original version of the model, simulations were performed on a closed one dimensional chain. Agents can take only one of two possible opinions -1 or +1. In a simulation, groups of agents are arranged to meet, discuss and change current opinion. Such dynamics lead to a so-called consensus, i.e. a state in which all agents share the same opinion. There are two possible states in this model. First is social validation - when people share the same opinion, then their neighbors start to agree with them. Second is discord destroys - when connected individuals disagree, then their neighbors start to argue with them too.
This model, with modifications, has been applied to explain the process of opinion exchange in finance, politics, and marketing, among other fields \cite{sznajd_2005}.


\subsection{Threshold Models}
Threshold models describe how individuals decide to change their views or behaviour by following the behaviour of their friends \cite{granovetter_1978}. These models can be used to analyse a wide range of phenomena such as the spread of information, rumours and social influence. Individuals change their views when a certain percentage of their neighbours have done so too. Depending on the type of relationship and its importance, different neighbours may have a different role in convincing a particular individual to change his decision. \par
The most popular of the threshold models used for the social influence problem is the Linear Threshold Model \cite{granovetter_1978}. In this model, individuals are connected by relationships with different weights, that characterise the strength of the interaction between neighbours. Each of them is also assigned a threshold representing the boundary value beyond which a change of view occurs. The process ends when none of the individuals influences anyone else anymore. Threshold values assigned to specific individuals can be initialised by drawing from some probability distribution or initialised as fixed values.
Threshold models found applications in modeling diffusion of innovations \cite{valente_1996} and spread of social influence \cite{kempe_2003}.


\subsection{Binary Agreement Models}
The binary agreement model also called a naming game model is another approach to modelling the spread of opinions. The process is iterative. In each iteration, a speaker and a listener are drawn at random and linked by a relationship. Then they communicate with each other, exchanging their opinions. In this model, each individual is assigned one of two competing views or both opinions. During communication, if the listener already holds the view presented by the speaker, they both retain it and discard the opposing view. Otherwise, the listener updates their set of views with the newly presented one. \par
Research conducted by scientists has shown that this model is able to guarantee consensus for individuals organised in a complete graph structure. The time in which this happens depends on the number of agents in the network \cite{castello_baronchelli_vittorio_2009}. A very interesting property of this model is also the fact that individuals promoting their own views seek a broader consensus \cite{kearns_2009}. Using this model, it has been shown, that even a small proportion of individuals resistant to change of opinion and constantly proclaiming their fixed views, can be sufficient to convince an overwhelmingly large proportion of the community, to adopt a new view \cite{szymanski_2011}.


\subsection{Naive Learning Models}
% Naive learning
These models are based on the assumption, that individuals do not perform particularly complex calculations and analyses, when updating their opinions. Naive learning, also known as DeGroot learning, works on the principle that individuals periodically communicate with their neighbors and average their opinions in the simplest possible way. This myopic model is more advanced than simple models of information propagation but still uncomplicated. Despite this relative simplicity, it has proven capable of adequately modeling the behavior of real human communities. This model is considered to be one of the canonicals for representation of opinion propagation processes, due to its characteristics. An important property in this context is that once certain known assumptions are met, a consensus is guaranteed to be reached in a finite time \cite{banerjee_2019}. It has been widely applied in many fields such as economics, sociology, statistics, and politics \cite{demarzo_2003}. \par
% De Groot model
In the DeGroot model, agents are connected in a weighted directed network describing the directions of information flow between individuals. Agents start the learning process with some fixed views. Then, in an iterative process, each individual establishes its view as an average of its previous opinion and that of its neighbors. Over time, the opinions of all individuals align and, assuming certain criteria are met, the network reaches consensus \cite{jackson_2008}. \par
% Properties, other
This model simulates some phenomena characteristic of human communities in a very interesting way. The own opinion of individuals can act as an echo, and come back to them, increasing their confidence in the stated theory. Another example is the repeated counting of opinions of individuals sharing a larger number of common neighbors. The biggest influence on the spread of an individual's opinion is his centrality in the network structure.
The final consensus depends on the initial views and the value of the centrality measure of the nodes in the network. This model makes it extremely simple to compute the state of views in a network at any point in time, using matrix computation. The formula boils down to summing the initial opinions of each individual multiplied by its eigenvector centrality. The conditions for achieving consensus in the DeGroot model are a diversity of initial views that cannot be systematically biased and a balanced structure of connections in the network. The eigenvector centrality of each individual in the network should be relatively small, compared to the sum of the centralities of the rest of the agents. In other words, no small group of agents can have a monopoly on proclaiming their views \cite{jackson_golub_2010}. In social networks, where some individuals have too much influence on the rest of the community, due to a large number of contacts, it may be impossible to carry out a correct learning process. \par


\subsection{Observational Learning Models}
% Observational learning
Observational learning models are characterized by the fact, that individuals within the network structure can observe each other's actions and the results they achieve. This group of models is, by the nature of its operation, most similar to the philosophical theories of social choice and social epistemology. These theories focus on the issue of aggregating individual views into group beliefs. Working together is better than going it alone because when faced with a problem to solve, alone people tend to get attached to wrong theories too quickly. \par

% Network Epistemology Model
Representative of this category of models is the network epistemology model, which is based on ideas from many fields like biology and economics combined to explain learning in epistemic communities. Epistemic communities are special cases of social communities, whose members strive to gain knowledge about the world around them. An example of such a community might be a community of scientists in some field. Such communities of scientists have often been the focus of researches performed using this model, but it has a much broader application. It can be used to model any community of people gathering evidence from their environment and exchanging beliefs \cite{missinformation_age}.
This is because all people, both individually and in communities interact with the outside world in different ways. Depending on the results of these interactions, they also adjust their perspective on the world and their strategies of acting. It is in this way - trying to understand the world on the basis of past experience - that people act in a scientifically similar way, though certainly less precisely and systematically.
In this model, unlike simpler ones such as epidemic models, for example, beliefs do not simply transfer from one person to another. Each individual has some degree of confidence in a particular view, which leads him to collect evidence to support it. This evidence influences the subsequent verification of the views. Each agent shares the collected evidence with his neighbors, which also influences the formation of their opinions. \par

\begin{figure}
\centering\includegraphics[width=.8\textwidth]{img/rozdzial1/epinet_example.png}
\caption{An example of an opinion spread in an epistemic network. (Source: \cite{sep-epistemology-social})}  \label{rys:network}
\end{figure}

% Bala & Goyal
The network epistemology model was initially proposed in \cite{bala_goyal} to model the process of knowledge propagation. In this model, agents are connected in an undirected network. The learning process is iterative. The problem that agents face includes choosing between two options. Agents have certain beliefs about which action is better. At the beginning of the simulation, agents are assigned random confidence levels about the effectiveness of particular actions. In each iteration, the agents choose the action about which they have the highest confidence in its success. The agents do not know the expected payoff value of the action. Their knowledge is based only on observing the results returned by the actions, which are random in nature. After executing the chosen action, individuals update their views using Bayes' rule based on their own and their neighbors' observations. The simulation ends when the network converges to one of the views. In this most basic version of the model, it means that the communities in question come to a correct or incorrect consensus.
Using this model, it has been observed that even a community of perfectly rational people, trying to figure out the truth about the world, can come to false conclusions. This is because the actual scientific evidence obtained by interacting with the world is probabilistic. For example, not all cigarette smokers will develop lung cancer. By observing too small a control group, it may be that none of the smokers get sick. Such cases, though obviously less likely, do occur and are the genesis of studies showing no statistical relationship between smoking and cancer. Such a misleading study, if sufficiently well publicized, may be enough to mislead the entire community. \par

% Zollman
Subsequently, work on this model was undertaken by philosopher Kevin Zollman. It was he who proposed using the model to mimic the operation of scientific communities. His work shows that the model can help understand the consensus building process in such communities. At first glance, it would seem that better communication of experimental results between individuals should improve community learning. It is true that such more densely connected communities come to a consensus more quickly, but more often than not it is an incorrect consensus. The results of his simulations show that the structure of a social network has a direct impact on the likelihood of reaching a correct consensus \cite{zollman_2011}. In particular circumstances, fewer connections and thus a slower process of exchanging views may help the whole community to reach correct conclusions. In a more precise study, it was later shown that this effect only occurs if learning is difficult and agents have great difficulty distinguishing between better theory \cite{oconnor_rosenstock_2017}. \\
The difficulty of the problem in this model is based on its three parameters:
\begin{itemize}
    \item Payoffs of the two possible actions is very similar.
    \item The community size is small.
    \item The number of evidence collected by individuals in each iteration is small.
\end{itemize} \par

% O'Connor & Weatherall
In the last few years, there have been proposed some interesting extensions to the model proposed by O'Connor and Weatherall, that make it even better at imitating the actual process of opinion spreading in social networks. \par
% Conformity
Conformity is the preferred choice to behave the way the rest of the members of a community do. This drive for conformity is deeply ingrained in human psychology. When conformism is added to the model, cliques of agents who insist on erroneous beliefs emerge. The phenomenon of conformity emergence makes it more difficult for the community to reach a correct consensus \cite{oconnor_weatherall_2018}. \par
% Polarization
If to this model we add social capital of trust or a tendency to conformism we find that given groups are no longer able to reach consensus. When each individual trusts the evidence of those who share his or her views, then extreme dissenting camps form in which everyone listens only to their own. When each individual tries to conform his behavior to the rest of his group, then good ideas are unable to spread between groups.
Social trust matters when an individual treats some sources of information with more respect than others. We can observe this phenomenon when agents trust more evidence provided by members of the same group. Polarization is a phenomenon that occurs frequently in societies. Certain closed subgroups hold a given opinion despite the fact that it is not true. The process behind this is based on unequal treatment of evidence presented by individuals with differing views \cite{oconnor_weatherall_2018_3}. \par
% Propaganda
Industry propagandists shape public opinion by selectively sharing only those results that insidiously support worse behavior. This can also mislead audiences when a group of evidence seekers reaches a consensus by coming to the wrong conclusion. Such a strategy used for public disinformation exploits the natural potential for the randomness of scientific results to mislead. Accounting for such behavior requires the introduction of new classes of nodes into the model, with behavior different from the default - scientific. Agents who are propagandists may share false results, or deliberately publish only the part of their results that discredits a view, and hide those that support it. A small fraction of propagandists can completely sabotage a community's learning process \cite{oconnor_weatherall_2020_1}, \cite{oconnor_weatherall_2020_2}. \par
% endogenous epistemic fractionalization
The problem of epistemic fractionalization was also studied using this model. This is the situation in which people who disagree on one issue tend to disagree on others as well. This related mistrust leads to the endogenous emergence of factions of individuals in the community with highly correlated views on a wide variety of topics. This behavior explains the actual tendency of people to accept full packages of views from the groups to which they belong \cite{oconnor_weatherall_2018_2}.


\section{Summary}
The overview of methods used to model the spread of views presented in this chapter provides an overview of the subject and an understanding of the variety of approaches and challenges faced by researchers working in this area. Different approaches presented, focusing on a macroscopic view of the problem and a microscopic focus on the individual. \par
One of the more advanced and recently focused approaches is the network epistemology model, which covers important concepts that are missing from simpler models. The fact that it captures many social phenomena such as conformism, polarization, and the influence of propaganda allows it to be widely applied. Also, the nature of the model itself and the process of working proposed in it are extremely interesting because of its similarity to the psychological and cognitive processes used by humans on a daily basis. It is this model that the later part of the work will focus on, beginning in the third chapter.
